\chapter{Конструкторская часть}

\section{Описание используемых структур}
При реализации алгоритмов будут использованы нижеприведенные структуры.
\begin{itemize}
	\item Заявка --- обрабатываемый элемент, хранящий время входа и выхода с каждого конвейера, время обработки и обрабатываемый элемент (символ).
	\item Очередь ---  основной компонент конвейера. Здесь храняться элементы в ожидании обработки.
	\item Конвейер ---  реализует обработку элементов, хранение очереди, может манипулировать элементами.
\end{itemize} 

\section{Описание алгоритмов}
Всего будут создано три конвейера. Все контейнеры могут работать либо параллельно либо последовательно.
На рисунках \ref{ris:NRL1}, \ref{ris:NRL2} -- \ref{ris:NRL5} приведены схемы линейной и параллельной обработки соответственно.  
\begin{center}	
\newpage
	\centering
	\def\svgwidth{12cm}
	\input{src/Linear.pdf_tex}
	\captionof{figure}{Схема линейной конвейерной обработки}
	\label{ris:NRL1}
\end{center}
\begin{center}	
	\centering
	\def\svgwidth{5cm}
	\input{src/Parallel1.pdf_tex}
	\captionof{figure}{Схема параллельной конвейерной обработки}
	\label{ris:NRL2}
\end{center}
\begin{center}	
	\newpage
	\centering
	\def\svgwidth{11cm}
	\input{src/Parallel2.pdf_tex}
	\captionof{figure}{Схема 1 потока параллельной конвейерной обработки}
	\label{ris:NRL3}
\end{center}
\begin{center}	
	\centering
	\def\svgwidth{11cm}
	\input{src/Parallel3.pdf_tex}
	\captionof{figure}{Схема 2 потока параллельной конвейерной обработки}
	\label{ris:NRL4}
\end{center}
\begin{center}	
	\newpage
	\centering
	\def\svgwidth{11cm}
	\input{src/Parallel4.pdf_tex}
	\captionof{figure}{Схема 3 потока параллельной конвейерной обработки}
	\label{ris:NRL5}
\end{center}

\section*{Вывод}
В данном разделе --- были описаны используемые структуры и приведены схемы алгоритмов.