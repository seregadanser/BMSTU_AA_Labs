\chapter{Конструкторская часть}

\section{Описание алгоритмов}
В данном разделе будут рассмотрены схемы алгоритмов поиска расстояний --- Левенштейна (рис. \ref{ris:NRL1}, \ref{ris:NRL2}), Дамерау-Левенштейна матричным способом (рис. \ref{ris:NRDL1}, \ref{ris:NRDL2}), Дамерау-Левенштейна рекурсивным способом (рис. \ref{ris:RDL}), Дамерау-Левенштейна рекурсивным с кеширование способом (рис. \ref{ris:СDL1}, \ref{ris:СDL2}).
\begin{center}
	

\newpage
	\centering
	\def\svgwidth{14cm}
	\input{src/NRL1.pdf_tex}
	\captionof{figure}{Схема алгоритма поиска р. Л. матричным способом, часть 1}
	\label{ris:NRL1}
\newpage

\centering
\def\svgwidth{14cm}
\input{src/NRL2.pdf_tex}
\captionof{figure}{Схема алгоритма поиска р. Л. матричным способом, часть 2}
\label{ris:NRL2}
\newpage

\centering
\def\svgwidth{14cm}
\input{src/NRDL1.pdf_tex}
\captionof{figure}{Схема алгоритма поиска р. Д-Л. матричным способом, часть 1}
\label{ris:NRDL1}
\newpage

\centering
\def\svgwidth{14cm}
\input{src/NRDL2.pdf_tex}
\captionof{figure}{Схема алгоритма поиска р. Д-Л. матричным способом, часть 2}
\label{ris:NRDL2}
\newpage

\centering
\def\svgwidth{12cm}
\input{src/RDL.pdf_tex}
\captionof{figure}{Схема алгоритма поиска р. Д-Л. рекурсивным способом}
\label{ris:RDL}
\newpage

\centering
\def\svgwidth{14cm}
\input{src/CDL1.pdf_tex}
\captionof{figure}{Схема алгоритма поиска р. Д-Л. рекурсивным способом с кешем, часть 1}
\label{ris:СDL1}
\newpage

\centering
\def\svgwidth{11cm}
\input{src/CDL2.pdf_tex}
\captionof{figure}{Схема алгоритма поиска р. Д-Л. рекурсивным способом с кешем, часть 2}
\label{ris:СDL2}
\newpage

\end{center}


\section{Использование памяти}
Замеры времени работы и используемой памяти алгоритмов Левенштейна и Дамерау-Левенштейна могут быть произведены одним и тем же способом.
Тогда рассмотрим только рекурсивную и матричную реализации данных алгоритмов.

Пусть n --- длина строки S1, m --- длина строки S2.

Тогда можно рассчитать затраты по памяти.
\begin{itemize}
	\item Алгоритм нахождения расстояния Левенштейна (матричный):
	\begin{itemize}
		\item для матрицы --- ((n + 1)$ \cdot$ (m + 1)) $ \cdot$ sizeof(int));
		\item для S1, S2 --- (n + m) $ \cdot$ sizeof(char);
		\item для n, m --- 2 $ \cdot$ sizeof(int);
		\item доп. переменные --- 6 $ \cdot$ sizeof(int);
		\item адрес возврата.
	\end{itemize}
	
	\item Алгоритм нахождения расстояния Дамерау-Левенштейна (матричный):
	\begin{itemize}
		\item для матрицы --- ((n + 1) $ \cdot$ (m + 1)) $ \cdot$ sizeof(int));
		\item для S1, S2 --- (n + m) $ \cdot$ sizeof(char);
		\item для n, m --- 2 $ \cdot$ sizeof(int);
		\item доп. переменные --- 7 $ \cdot$ sizeof(int);
		\item адрес возврата.
	\end{itemize}
	
	\item Алгоритм нахождения расстояния Дамерау-Левенштейна (рекурсивный), где для каждого вызова:
	\begin{itemize}
		\item для S1, S2 --- (n + m) $ \cdot$ sizeof(char);
		\item для n, m --- 2 $ \cdot$ sizeof(int);
		\item доп. переменные --- 5 $ \cdot$ sizeof(int);
		\item адрес возврата.
	\end{itemize}
	\newpage
	\item Алгоритм нахождения расстояния Дамерау-Левенштейна с использованием кеша в виде матрицы (память на саму матрицу: ((n + 1) $ \cdot$ (m + 1)) $ \cdot$ sizeof(int)) (рекурсивный), где для каждого вызова:
	\begin{itemize}
		\item для S1, S2 --- (n + m) $ \cdot$ sizeof(char);
		\item для n, m --- 2 $ \cdot$ sizeof(int);
		\item доп. переменные --- 5 $ \cdot$ sizeof(int);
		\item указатель на матрицу --- sizeof(int**);
		\item адрес возврата.
	\end{itemize}
	
\end{itemize}



\section*{Вывод}
В данном разделе --- были описаны алгоритма поиска расстояний Левенштейна и Дамерау-Левенштейна, рассчитана паямять, используемая данными алгоритмами.