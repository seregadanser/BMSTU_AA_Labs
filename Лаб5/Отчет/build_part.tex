\chapter{Конструкторская часть}

\section{Описание используемых структур}
При реализации алгоритмов будут использованы нижеприведенные структуры.
\begin{itemize}[itemindent=1.25em]
	\item[---] Заявка --- обрабатываемый элемент, хранящий время входа и выхода с каждого конвейера, состояние заявки до и после обработки, время обработки и обрабатываемый элемент (строка). Также в структуре заявки находится логический флаг, для проверки является ли эта заявка последней.
	\item[---] Очередь ---  основной компонент конвейера и линейной обработки. Здесь хранятся элементы в ожидании обработки. Очередь является разделяемым ресурсом: с ней работают различные потоки.
	\item[---] Конвейер ---  реализует обработку элементов, хранение очереди, может манипулировать элементами.
\end{itemize} 

\section{Описание алгоритмов}
Всего будет создано три конвейерные ленты.

На рисунках \ref{ris:NRL1}, \ref{ris:NRL2} -- \ref{ris:NRL5} приведены схемы линейной и конвейерной обработки соответственно. Самой последней в начальную очередь добавляется особая заявка, у которой поднят флаг, отвечающий за то, то она последняя. При работе алгоитмов все заявки проходят проверку на то поднят данный флаг или нет, при положительном результате текущий этап считается завершенным
\begin{center}	
	\centering
	\def\svgwidth{12cm}
	\input{src/Linear.pdf_tex}
	\captionof{figure}{Схема линейной обработки}
	\label{ris:NRL1}
\end{center}
\begin{center}	
	\centering
	\def\svgwidth{5cm}
	\input{src/Parallel1.pdf_tex}
	\captionof{figure}{Схема параллельной конвейерной обработки}
	\label{ris:NRL2}
\end{center}
\begin{center}	
	\newpage
	\centering
	\def\svgwidth{11cm}
	\input{src/Parallel2.pdf_tex}
	\captionof{figure}{Схема 1 потока параллельной конвейерной обработки}
	\label{ris:NRL3}
\end{center}
\begin{center}	
	\centering
	\def\svgwidth{11cm}
	\input{src/Parallel4.pdf_tex}
	\captionof{figure}{Схема 2 потока параллельной конвейерной обработки}
	\label{ris:NRL4}
\end{center}
\begin{center}	
	\centering
	\def\svgwidth{11cm}
	\input{src/Parallel3.pdf_tex}
	\captionof{figure}{Схема 3 потока параллельной конвейерной обработки}
	\label{ris:NRL5}
\end{center}

\section*{Вывод}
В данном разделе были описаны используемые структуры и приведены схемы алгоритмов.