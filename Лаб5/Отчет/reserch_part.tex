\chapter{Исследовательская часть}

В данном разделе будут приведены примеры работы программы, а также проведен сравнительный анализ процессорного времени работы реализаций алгоритмов при различных ситуациях на основе полученных данных.

\section{Технические характеристики}

Технические характеристики устройства, на котором выполнялись замеры времени представлены далее:

\begin{itemize}
	\item операционная система Windows 11 Pro Версия 22H2 (22621.674) \cite{wind};
	\item память 16 ГБ;
	\item процессор 11th Gen Intel(R) Core(TM) i5-11400 2.59 ГГц \cite{proc}.
\end{itemize}

При тестировании компьютер был включен в сеть электропитания. Во время замеров процессорного времени устройство было нагружено только встроенными приложениями окружения, а также системой тестирования.

\section{Демонстрация работы программы}

На рисунке \ref{img:res} представлен результат работы программы. Работающая программа выводит на экран журнал отладки для каждой заявки.
%\newpage
\begin{center}
	\centering{\includegraphics[trim=0 0 0cm 0cm bb=0 0 504 393]{src/screen}}
	\captionof{figure}{Пример работы программы}
	\label{img:res}
\end{center}

\section{Время выполнения реализаций алгоритмов}

\textbf{Входные данные:} строки размером от 1 до 100 символов.

В таблице ниже приведены результаты замеров времени (в тиках) для общего времени пребывания заявок в очередях.

\begin{center}
	\begin{threeparttable}
		\caption{Суммарное время пребывания всех заявок в очереди}
		\label{tbl:best}
		\begin{tabular}{|c|c|c|}
			\hline
			Входная строка &Линейный конвейер &Параллельный конвейер\\
			\hline
			a& 151 &81\\
			\hline
			zdl &829 & 256\\
			\hline
			qawsed&782 &370 \\
			\hline
			asdfghjoiujy&11440 &2483 \\
			\hline
		\end{tabular}
		
	\end{threeparttable}
\end{center}


 На рисунке \ref{img:graph_sorted} приведены графические результаты замеров времени работы алгоритмов в зависимости от линейного размера входной строки.

\begin{center}
	\centering{\includegraphics[trim=0 0 0 -5cm bb=0 0 1300 1000]{src/good}}
	\captionof{figure}{Время вычислений}
	\label{img:graph_sorted}
\end{center}





\section{Вывод}
По полученным результатам ислледования можно сделать вывод, что параллельный конвейер выигрывыает от двух до трех раз в скорости работы и снижении времени ожидания заявок в очереди.
