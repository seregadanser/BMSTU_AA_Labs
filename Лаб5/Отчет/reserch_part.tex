\chapter{Исследовательская часть}

В данном разделе будут приведены примеры работы программы, а также проведен сравнительный анализ процессорного времени работы реализаций алгоритмов при различных ситуациях на основе полученных данных.

\section{Технические характеристики}

Технические характеристики устройства, на котором выполнялись замеры времени, представлены далее:

\begin{itemize}[itemindent=1.25em]
	\item[---] операционная система Windows 11 Pro Версия 22H2 (22621.674) \cite{wind};
	\item[---] память 16 ГБ;
	\item[---] процессор 11th Gen Intel(R) Core(TM) i5-11400 2.59 ГГц \cite{proc}, 6 физических и 12 логических ядер.
\end{itemize}

При тестировании компьютер был включен в сеть электропитания. Во время замеров процессорного времени устройство было нагружено только встроенными приложениями окружения, а также системой тестирования.

\section{Демонстрация работы программы}

На рисунке \ref{img:res} представлен результат работы программы. Работающая программа выводит на экран журнал отладки для каждой заявки.
%\newpage
\begin{center}
	\centering{\includegraphics[trim=0 0 0cm 0cm bb=0 0 1008 786, scale=0.75]{src/screen}}
	\captionof{figure}{Пример работы программы}
	\label{img:res}
\end{center}

\section{Время выполнения реализаций алгоритмов}

\textbf{Входные данные:} 1000 строк, размером от 10000 до 25000 символов.

В таблице \ref{tbl:best} приведены результаты замеров времени (в тиках) для общего времени пребывания заявок в очередях.

\begin{center}
	\begin{threeparttable}
		\caption{Суммарное время пребывания всех заявок в очереди в тиках $* 10^{-5}$}
		\label{tbl:best}
		\begin{tabular}{|c|c|c|}
			\hline
			Длина входных строк &Линейный конвейер &Параллельный конвейер\\
			\hline
			10000 & 33102498& 24064766 \\
			\hline
		    15000& 343765980 & 29704767 \\
		    \hline
		    20000& 454612686 & 375795895\\
			\hline
		    25000 & 536216748& 466675101 \\
			\hline
		\end{tabular}
		
	\end{threeparttable}
\end{center}


 На рисунке \ref{img:graph_sorted} приведены графические результаты замеров времени работы алгоритмов в зависимости от линейного размера входной строки.

\begin{center}
	\centering{\includegraphics[trim=0 0 0 -5cm bb=0 0 800 800, scale=0.75]{src/good}}
	\captionof{figure}{Время вычислений}
	\label{img:graph_sorted}
\end{center}





\section{Вывод}
По полученным результатам иследования можно сделать вывод, что есть разница между скоростью работы конвейерной обработки и последовательной обработки в полтора раза.
