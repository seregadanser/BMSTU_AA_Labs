\chapter{Технологическая часть}
В данном разделе будут рассмотрены средства реализации, а также представлены листинги алгоритмов.
\section{Требования к программному обеспечению }

Программа должна выводить журнал отладки для всех заявок и всех видов организации конвейерной обработки. 

\section{Средства реализации}
В данной работе для реализации был выбран язык программирования $c\#$. В текущей лабораторной работе требуется замерить время пребывания заявок в очередях. Для этого будет использована структура $DateTime$, предоставляющая текущее время. Использовать структуру приходится дважды, затем из конечного времени
нужно вычесть начальное, чтобы получить результат. Для визуализации результатов использовался язык $Python$.

\section{Сведения о модулях программы}
Программа состоит из следующих модулей:
\begin{itemize}
	\item класс $Conver$ реализует конвейер; 
	\item класс $Ask$ реализует заявку;
	\item класс $Crypto$ реализует алгоритмы шифрования;
	\item класс $Formatter$ реализует вывод состояний на экран.
\end{itemize}

\section{Реализация алгоритмов}
В листингах , представлены реализации алгоритмов. 
\begin{center}
\begin{lstlisting}[label=l1, caption={Реализация конвейера}]
class Conver
{
	ConsoleApp1.Action a;
	int id;
	public Queue<Ask> nextState;
	public Queue<Ask> queue;
	int key;
	public Conver(ConsoleApp1.Action a, int key, int id)
	{
		this.a = a;
		queue = new Queue<Ask>();
		this.key = key;
		this.id = id;
	}
	public Conver(ConsoleApp1.Action a, int key, Queue<Ask> q, int id) : this(a, key, id)
	{
		queue = q;
	}
	public void Start()
	{
		state s = state.ok;
		while(s!=state.finish)
		{
			s = Process();
		}
	}
	public state Process()
	{
		Ask elem = null;
		bool e = false;
		lock (queue)
		{
			if (queue.Count > 0)
			e = queue.TryDequeue(out elem);
		}
		if (e && elem.last)
		{
			elem.out_time[id] = DateTime.Now.Ticks;
			elem.in_time[id + 1] = DateTime.Now.Ticks;
			lock (nextState)
			{ nextState.Enqueue(elem); }
			return state.finish; 
		}
		if (e)
		{
			elem.out_time[id] = DateTime.Now.Ticks;
			long s = DateTime.Now.Ticks;            
			elem.elem = a.Invoke(elem.elem, key); 
			elem.work_time[id] = DateTime.Now.Ticks - s;
			elem.state[id+1] = elem.elem;
			elem.in_time[id + 1] = DateTime.Now.Ticks;
			lock (nextState)
			{ nextState.Enqueue(elem); }
		}
		return state.ok;
	}	
	public void StartSerial()
	{
		Ask elem = queue.Dequeue();
		elem.out_time[id] = DateTime.Now.Ticks;
		if (elem.last)
		return;
		long s = DateTime.Now.Ticks;
		elem.elem = a.Invoke(elem.elem, key);
		elem.work_time[id] = DateTime.Now.Ticks - s;
		elem.state[id + 1] = elem.elem;
		elem.in_time[id + 1] = DateTime.Now.Ticks;
		nextState.Enqueue(elem);
	}
}
\end{lstlisting}
\end{center}
\begin{center}
\begin{lstlisting}[label=l2, caption={Реализация линейной обработки данных}]
void SerialCode()
{
	Conver[] c = new Conver[3];
	c[0] = new Conver(new ConsoleApp1.Action(Crypto.CodeEncode),300, q, 0);
	c[1] = new Conver(new ConsoleApp1.Action(Crypto.Cipher), 400, 1);
	c[2] = new Conver(new ConsoleApp1.Action(Crypto.CodeEncode), 500, 2);
	c[0].nextState = c[1].queue;
	c[1].nextState = c[2].queue;
	c[2].nextState = exit;
	c[0].Start();
	c[1].Start();
	c[2].Start();
}
\end{lstlisting}
\end{center}
%\newpage
\begin{center}
\begin{lstlisting}[label=l3,caption = {Реализация параллельной обработки данных}]
	void ParalCode()
	{
		Thread[] t = new Thread[3];
		Conver[] c = new Conver[3];
		c[0] = new Conver(new ConsoleApp1.Action(Crypto.CodeEncode),300, q, 0);
		c[1] = new Conver(new ConsoleApp1.Action(Crypto.Cipher), 400, 1);
		c[2] = new Conver(new ConsoleApp1.Action(Crypto.CodeEncode), 500, 2);
		c[0].nextState = c[1].queue;
		c[1].nextState = c[2].queue;
		c[2].nextState = exit;
		for (int i = 0; i < 3; i++)
		{
			t[i] = new Thread(c[i].Start);
			t[i].Start();
		}
		foreach (Thread thread in t)
		{
			thread.Join();
		}
	}
\end{lstlisting}
\end{center}
\begin{center}
	\begin{lstlisting}[label=l4,caption = {Реализация алгоритмов шифрования}]
		 static class  Crypto
		{
			const string alfabet = "ABCDEFGHIJKLMNOPQRSTUVWXYZ";
			static public char CodeEncode(char text, int k)
			{
				var fullAlfabet = alfabet.ToLower();// + alfabet.ToLower();
				var letterQty = fullAlfabet.Length;
				char retVal = '\0';
				
				var index = fullAlfabet.IndexOf(text);
				if (index < 0)
				{
					retVal = text;
				}
				else
				{
					var codeIndex = (letterQty + index + k) % letterQty;
					retVal = fullAlfabet[codeIndex];
				}
				return retVal;
			}
			static char GetRandomKey(int k)
			{
				char gamma = '\0';
				var rnd = new Random(k);
				gamma = (char)rnd.Next(97, 123);
				return gamma;
			}
			static public char Cipher(char text, int key)
			{
				var currentKey = GetRandomKey(key);
				char res = '\0';
				res = ((char)(text ^ key));
				return res;
			}
		}
	\end{lstlisting}
\end{center}
\begin{center}
	\begin{lstlisting}[label=l5,caption = {Реализация структуры заявки}]
		class Ask
		{
			public long[] in_time, out_time, work_time;
			public char[] state;
			public char elem;
			public bool last;
			public Ask(int n, char elem, bool last = false)
			{
				in_time = new long[n];
				out_time = new long[n];
				work_time = new long[n];
				state = new char[n];
				this.elem = elem;
				this.last = last;
			}
		}
	\end{lstlisting}
\end{center}


\newpage
\section{Функциональное тестирование}

В таблице \ref{tbl:functional_test} приведены тесты для функций, реализующих алгоритмы конвейерной обработки. Тесты для всех реализаций алгоритмов пройдены успешно.
	\begin{center}

				\begin{threeparttable}
					\captionsetup{justification=raggedright,singlelinecheck=off}
					\caption{\label{tbl:functional_test} Функциональные тесты}
					\begin{tabular}{|c|c|}
						\hline
						Входная строка & Выходная строка \\
						\hline
						Пустая строка & Сообщение об ошибке\\
						\hline
						a & q\\
						\hline
					\end{tabular}
				\end{threeparttable}
	\end{center}



\section*{Вывод}
% \addcontentsline{toc}{section}{Вывод}
