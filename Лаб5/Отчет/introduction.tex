
\ssr{Введение}

Использование параллельной обработки уменьшает время выполнения программы. Конвейерная обработка является одним из примеров, где использование параллелизации помогает ускорить обработку данных. Суть та же, что и при работе реальных конвейерных лент ---
материал поступает на обработку, после окончания обработки материал передается на место следующего обработчика, при этом предыдыдущий
обработчик не ждёт полного цикла обработки материала, а получает новый
материал и работает с ним.

Целью данной лабораторной работы является изучение, реализация и исследование конвейерной обработки данных.

Для достижения поставленной цели требуется решить ряд задач:
\begin{enumerate}
	\item[1)] изучить конвейерную обработку;
	\item[2)] разработать конвейеры и алгоритмы, выполняемые на них;
	\item[3)] реализовать заданные алгоритмы и конвейеры;
	\item[4)] провести замеры времени для каждой из реализаций алгоритмов;
 	\item[5)] выполнить анализ полученных результатов;
	\item[6)] по итогам работы составить отчет.
\end{enumerate}

\newpage
