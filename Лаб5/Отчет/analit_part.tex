\chapter{Аналитическая часть}

В данном разделе будут рассмотрены главные принципы конвейерной обработки и параллельного доступа к данным.

\section{Конвейерная обработка данных}
Конвейеризация (или конвейерная обработка) --- это такая организация выполнения операций над объектами, при которой весь процесс воздействия разделяется на последовательность стадий с целью повышения производительности путём одновременного независимого выполнения операций над несколькими объектами, проходящими различные стадии. Конвейером также называют средство продвижения объектов между стадиями при такой организации.

\section{Последовательная обработка данных}
При последовательной обработке данных заявки попадают в очереди последовательно друг за другом. Пока одна заявка не будет обработана, другие заявки не смогут попасть в какие-либо очереди и вынуждены ждать.

\section{Описание используемых алгоритмов}
В качестве примера для операции, подвергающейся конвейерной обработке, будет обрабатываться массив строка, разделенный на строки. 
\newpage
В программе будут использованы следующие алгоритмы:
\begin{itemize}
	\item шифровка входной строки шифром Цезаря;
\item шифровка входной строки $xor$ шифром;
\item шифровка входной строки шифром Цезаря.
\end{itemize} 

Шифр Цезаря --- это вид шифра подстановки, в котором каждый символ в открытом тексте заменяется символом, находящимся на некотором постоянном числе позиций левее или правее него в алфавите.

Шифр XOR --- это алгоритм шифрования данных с использованием исключительной дизъюнкции.

\section{Организация взаимодействия параллельных потоков}
Потоки исполняются в общем адресном пространстве программы. Как результат, взаимодействие параллельных потоков можно организовать через использование общих данных, являющихся доступными для всех потоков. Наиболее простая ситуация состоит в использовании общих данных только для чтения. В случае же, когда общие данные могут изменяться несколькими потоками, необходимо блокировать участки кода, в которых происходит запись или чтение в общую переменную. 
\section*{Вывод}
В данном разделе были рассмотрены основы конвейерной обработки, алгоритмы, которые лягут в основу конвейеров и
организация взаимодействия параллельных потоков.