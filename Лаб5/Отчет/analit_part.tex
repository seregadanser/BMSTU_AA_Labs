\chapter{Аналитическая часть}

В данном разделе будут рассмотрены главные принципы конвейерной обработки и параллельного доступа к данным.

\section{Конвейерная обработка данных}
Конвейеризация (или конвейерная обработка) --- это такая организация выполнения операций над объектами, при которой весь процесс воздействия разделяется на последовательность стадий с целью повышения производительности путём одновременного независимого выполнения операций над несколькими объектами, проходящими различные стадии. Конвейером также называют средство продвижения объектов между стадиями при такой организации.

Обработку любой операции можно разделить на
несколько стадий, организовав передачу данных от одной стадии к следующей. При этом конвейерную обработку можно использовать для совмещения этапов выполнения разных операций. Производительность при этом возрастает, благодаря тому, что одновременно на различных
ступенях конвейера выполняется несколько задач. Конвейерная обработка
такого рода широко применяется во всех современных быстродействующих
процессорах.
Конвейеризация повзволяет увеличить пропускную способность процессора (количество команд, завершающихся в единицу времени), но она не сокращает время выполнения отдельной команды. Увеличение пропускной способности означает, что программа будет выполняться
быстрее по сравнению с простой, неконвейерной схемой.

\section{Описание используемых алгоритмов}
В качестве примера для операции, подвергающейся конвейерной обработке, будет обрабатываться строка. 
В программе будет использовано три конвейера, выполняющих следующие опреации:
\begin{itemize}
	\item шифровка входной строки шифром цезаря;
\item шифровка входной строки $xor$ шифром;
\item шифровка входной строки шифром цезаря;
\end{itemize} 

Шифр Цезаря --- это вид шифра подстановки, в котором каждый символ в открытом тексте заменяется символом, находящимся на некотором постоянном числе позиций левее или правее него в алфавите.

Шифр XOR --- это алгоритм шифрования данных с использованием исключительной дизъюнкции.

\section{Организация взаимодействия параллельных потоков}
Потоки исполняются в общем адресном пространстве программы. Как результат, взаимодействие параллельных потоков можно организовать через использование общих данных, являющихся доступными для всех потоков. Наиболее простая ситуация состоит в использовании общих данных только для чтения. В случае же, когда общие данные могут изменяться несколькими потоками, необходимо блокировать участки кода, в которых происходит запись или чтение в общую переменную. 
\section*{Вывод}
В данном разделе были рассмотрены основы конвейерной обработки, алгоритмы, которые лягут в основу конвейеров и
организация взаимодействия параллельных потоков.