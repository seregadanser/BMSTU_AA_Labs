
\ssr{Введение}

Операции работы со строками являются важными компонентами в программирования. Часто возникает потребность в использовании строк при решении различных
задач, в которых нужны алгоритмы сравнения строк, о которых
и пойдет речь в данной работе. Одними из самых популярных алгоритмов в данной сфере являютя алгоритмы нахождения расстояний Левенштейна и Дамерау-Левенштейна.

Расстояние Левенштейна --– минимальное количество редакционных операций (вставка, удаление, замена символа), необходимых для преобразования одной строки в другую. 

Если текст был набран с клавиатуры, то вместо расстояния Левенштейна чаще используют расстояние Дамерау-Левенштейна, в котором добавляется еще одно возможное действие --- перестановка двух соседних символов.

Расстояния Левенштейна и Дамерау-Левенштейна применяются в таких сферах, как: 
\begin{itemize}
	\item компьютерная лингвистика (автозамена в посиковых запросах, текстовая редактура);
	\item биоинформатика (последовательности белков);
	\item нечеткий поиск записей в базах (борьба с мошенниками и опечатками).
\end{itemize}

Целью данной лабораторной работы является изучение, реализация и исследование
алгоритмов нахождения расстояний Левенштейна и Дамерау-Левенштейна.
\newpage
Для достижения поставленной цели требуется решить ряд задач:
\begin{enumerate}
	\item[1)] изучить алгоритмы нахождения расстояний Левенштейна и Дамерау-Левенштейна;
	\item[2)] разработать алгоритмы поиска этих расстояний;
	\item[3)] расчитать затрачиваемую реализованными алгоритмами память;
	\item[4)] реализовать каждый из данных алгоритмов;
	\item[5)] провести замеры процессорного времени для каждой из реализаций алгоритмов;
 	\item[6)] выполнить анализ полученных результатов;
	\item[7)] по итогам работы составить отчет.
\end{enumerate}

\newpage
