\chapter*{Заключение}
\addcontentsline{toc}{chapter}{Заключение}
Цель, которая была поставлена в начале лабораторной работы, была достигнута: изучены, реализованы и исследованы
алгоритмы нахождения расстояний Левенштейна и Дамерау-Левенштейна.

В ходе выполнения лабораторной работы были решены все задачи:
\begin{enumerate}
	\item[1)] изучены алгоритмы нахождения расстояний Левенштейна и Дамерау-Левенштейна;
	\item[2)] разработаны алгоритмы поиска этих расстояний;
	\item[3)] расчитана затрачиваемая реализованными алгоритмами память;
	\item[4)] реализован каждый из данных алгоритмов;
	\item[5)] проведены замеры процессорного времени для каждой из реализаций алгоритмов;
	\item[6)] выполнен анализ полученных результатов;
	\item[7)] по итогам работы составлен отчет.
\end{enumerate}

В ходе проделанной работы было выявлено, что реализации рекурсивных алгоритмов поиска расстояний левенштейна и дамерау-Левенштейна требуют больших затрат по времени. Однако рекурсивный алгоритм без использования кеша поиска расстояния Дамерау-Левенштейна требует меньших затрат по памяти. 