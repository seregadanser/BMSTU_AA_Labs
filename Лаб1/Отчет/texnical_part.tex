\chapter{Технологическая часть}
В данном разделе будут рассмотрены средства реализации, а также представлены листинги сортировок.
\section{Требования к программному обеспечению }

Программа должна выводить процессорное время, затрачиваемое каждым из алгоритмов.

\section{Средства реализации}
В данной работе для реализации был выбран язык программирования $c++$. В текущей лабораторной работе требуется замерить процессорное время для выполняемой программы. Для визуализации результатов использовался язык $Python$.

Время работы реализаций алгоритмов было замерено с помощью функции \textit{GetProcessTimes(...)} \cite{time} из библиотеки $Windows.h$. Функция возвращает пользовательское процессорное временя типа float.
Использовать функцию приходится дважды, затем из конечного времени нужно вычесть начальное, чтобы получить результат.


\section{Сведения о модулях программы}
Программа состоит из следующих модулей:
\begin{itemize}
	\item $AA2.cpp$ --- файл, содержащий весь служебный код;
	\item $Matrix.cpp$ --- файл, реализующий операции над матрицами;
	\item $algoc.cpp$ --- файл, содержаший реализации алгритмов поиска расстояний Л. и Д-Л.;
	\item $TimeCompare.cpp$ --- файл, производящий замеры времени;
	\item $GetCPUTime.cpp$ --- файл, определяющий функцию замера времени;
	\item $Gen.cpp$ --- файл, генерирующий входные строки;
	\item $Matrix.hpp$ --- заголовочный файл модуля $Matrix.cpp$;
	\item $TimeCompare.hpp$ --- заголовочный файл модуля $TimeCompare.cpp$;
	\item $algoc.hpp$ --- заголовочный файл модуля $algos.cpp$;
	\item $GetCPUTime.hpp$ --- заголовочный файл модуля $GetCPUTime.cpp$;
	\item $Gen.hpp$ --- заголовочный файл модуля $Gen.cpp$.
\end{itemize}

\section{Реализация алгоритмов}
В листингах \ref{lnr} --  \ref{dlrc}, представлены реализации алгоритмов поиска расстояний --- Левенштейна и Дамерау-Левенштейна. 
\begin{center}
\begin{lstlisting}[label=lnr, caption={Матричный алгоритм поиска пути Левенштейна}]
int non_rec_lev(const char* s1, const char* s2)
{
	int n = strlen(s1) + 1;
	int m = strlen(s2) + 1;
	if (n == 1)
	return m - 1;
	if (m == 1)
	return n - 1;	
	data_t mtrx;
	mtrx.matrix = create_matrix(strlen(s1) + 1, strlen(s2) + 1);
	mtrx.n = strlen(s1) + 1;
	mtrx.m = strlen(s2) + 1;
	for (int i = 0; i < n; ++i)
	{	
		for (int j = 0; j < m; ++j)
		{
			
			if (i == 0 && j == 0)
			mtrx.matrix[i][j] = 0;
			else if (i == 0)
			mtrx.matrix[i][j] = j;
			else if (j == 0)
			mtrx.matrix[i][j] = i;
			else
			mtrx.matrix[i][j] = min_int(3, mtrx.matrix[i][j - 1] + 1,mtrx.matrix[i - 1][j] + 1,
			mtrx.matrix[i - 1][j - 1] + (s1[i - 1] != s2[j - 1])
			);
		}
	}
	int result = mtrx.matrix[n - 1][m - 1];
	free_matrix(&mtrx);
	return result;
}
\end{lstlisting}
\end{center}
\begin{center}
\begin{lstlisting}[label=dlnr, caption={Матричный алгоритм поиска пути  Дамерау-Левенштейна}]
int non_rec_dam_lev(const char* s1, const char* s2)
{
	int n = strlen(s1) + 1;
	int m = strlen(s2) + 1;
	if (n == 1)
	return m - 1;
	if (m == 1)
	return n - 1;
	data_t mtrx;
	mtrx.matrix = create_matrix(strlen(s1) + 1, strlen(s2) + 1);
	mtrx.n = strlen(s1) + 1;
	mtrx.m = strlen(s2) + 1;
	for (int i = 0; i < n; ++i)
	{
		for (int j = 0; j < m; ++j)
		{
			if (i == 0 && j == 0)
			mtrx.matrix[i][j] = 0;
			else if (i == 0)
			mtrx.matrix[i][j] = j;
			else if (j == 0)
			mtrx.matrix[i][j] = i;
			else
			{
				mtrx.matrix[i][j] = min_int(3,
				mtrx.matrix[i][j - 1] + 1,
				mtrx.matrix[i - 1][j] + 1,
				mtrx.matrix[i - 1][j - 1] + (s1[i - 1] != s2[j - 1])
				);
				if (i > 1 && j > 1 && s1[i - 1] == s2[j - 2] && s1[i - 2] == s2[j - 1])
				mtrx.matrix[i][j] = min_int(2, mtrx.matrix[i][j], mtrx.matrix[i - 2][j - 2] + 1);
			}
		}
	}
	int result = mtrx.matrix[n - 1][m - 1];
	free_matrix(&mtrx);
	return result;
}
\end{lstlisting}
\end{center}
%\newpage
\begin{center}
\begin{lstlisting}[label=dlr,caption = {Рекурсивый алгоритм поиска пути  Дамерау-Левенштейна}]
int rec_dam_lev(const char* s1, const char* s2)
{
	return rec_dam_lev1(s1, s2, strlen(s1), strlen(s2));
}

int rec_dam_lev1(const char* s1, const char* s2, const int li1, const int li2)
{
	int result = -1;
	if (li1 == 0 || li2 == 0)
	return std::abs(li1 - li2);
	result = min_int(3,
	rec_dam_lev1(s1, s2, li1 - 1, li2) + 1,rec_dam_lev1(s1, s2, li1, li2 - 1) + 1, 
	rec_dam_lev1(s1, s2, li1 - 1, li2 - 1) + (s1[li1 - 1] != s2[li2 - 1]));
	if (li1 > 1 && li2 > 1 && s1[li1 - 1] == s2[li2 - 2] && s1[li1 - 2] == s2[li2 - 1])
	result = min_int(2,result,rec_dam_lev1(s1, s2, li1 - 2, li2 - 2) + 1);
	return result;
}

\end{lstlisting}

\end{center}
\begin{center}
	\begin{lstlisting}[label=dlrc,caption = {Рекурсивый с кешем алгоритм поиска пути  Дамерау-Левенштейна}]
	int cache_dam_lev(const char* s1, const char* s2)
	{
		data_t cache;
		cache.matrix = create_matrix(strlen(s1) + 1, strlen(s2) + 1);
		cache.n = strlen(s1) + 1;
		cache.m = strlen(s2) + 1;
		fill_mtrx_with_inf(&cache);
		int r = cache_dam_lev1(&cache, s1, s2, strlen(s1), strlen(s2));
		free_matrix(&cache);
		return r;
	}
	
	int cache_dam_lev1(data_t* cache, const char* s1, const char* s2, const int li1, const int li2)
	{
		if (cache->matrix[li1][li2] != LONG_MAX)
		return cache->matrix[li1][li2];
		if (li1 == 0 && li2 == 0)
		{
			cache->matrix[li1][li2] = 0;
			return cache->matrix[li1][li2];
		}
		if (li1 == 0 && li2 > 0)
		{
			cache->matrix[li1][li2] = li2;
			return li2;
		}
		if (li2 == 0 && li1 > 0)
		{
			cache->matrix[li1][li2] = li1;
			return li1;
		}	
		int r1 = 0, r2 = 0, r3 = 0;
		r1 = cache_dam_lev1(cache, s1, s2, li1 - 1, li2) + 1;
		r2 = cache_dam_lev1(cache, s1, s2, li1, li2 - 1) + 1;
		r3 = cache_dam_lev1(cache, s1, s2, li1 - 1, li2 - 1) + (s1[li1 - 1] != s2[li2 - 1]);
		cache->matrix[li1][li2] = min_int(3, r1, r2, r3);
		int result = 0;
		if (li1 > 1 && li2 > 1 && s1[li1 - 1] == s2[li2 - 2] && s1[li1 - 2] == s2[li2 - 1])
		{
			int r4 = 0;		
			r4 = cache_dam_lev1(cache, s1, s2, li1 - 2, li2 - 2) + 1;	
			cache->matrix[li1][li2] = min_int(2, cache->matrix[li1][li2], r4);
		}
		return cache->matrix[li1][li2];
	}	
	\end{lstlisting}
	
\end{center}
\newpage
\section{Функциональное тестирование}

В таблице \ref{tbl:functional_test} приведены тесты для функций, реализующих алгоритмы поиска пути Левенштейна и Дамерау-Левенштейна. Тесты для всех реализаций алгоритмов пройдены успешно.
	\begin{center}

				\begin{threeparttable}
					\captionsetup{justification=raggedright,singlelinecheck=off}
					\caption{\label{tbl:functional_test} Функциональные тесты}
					\begin{tabular}{|c|c|c|c|c|}
						\hline
						& \multicolumn{2}{c|}{Входные данные}  & \multicolumn{2}{c|}{Ожидаемый результат} \\
						\hline
						№&Строка 1&Строка 2&Левенштейн&Дамерау-Л. \\
						\hline
						1&"пустая строка"&"пустая строка"&0&0 \\
						\hline
						2&"пустая строка"&слово&5&5 \\
						\hline
						3&проверка&"пустая строка"&8&8 \\
						\hline
						4&ремонт&емонт&1&1 \\
						\hline
						5&гигиена&иена&3&3 \\
						\hline
						6&нисан&автоваз&6&6 \\
						\hline
						7&спасибо&пожалуйста&9&9 \\
						\hline
						8&что&кто&1&1 \\
						\hline
						9&ты&тыква&3&3 \\
						\hline
						10&есть&кушать&4&4 \\
						\hline
						11&abba&baab&3&2 \\
						\hline
						12&abcba&bacab&4&2 \\
						\hline
					\end{tabular}
				\end{threeparttable}


	\end{center}



\section*{Вывод}
% \addcontentsline{toc}{section}{Вывод}
В данном разделе были представлены реализации следующих алгоритмов поиска расстояний: Левенштейна матричного, Дамерау-Левенштейна матричного, Дамерау-Левенштейна рекурсивного, Дамерау-Левенштейна рекурсивного с кешерованием. Выполнено тестирование реализаций алгоритмов.