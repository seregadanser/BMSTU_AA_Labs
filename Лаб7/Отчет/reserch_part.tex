\chapter{Исследовательская часть}

В данном разделе будет проведен сравнительный анализ времени работы реализаций алгоритмов при различных ситуациях на основе полученных данных.

\section{Технические характеристики}

Технические характеристики устройства, на котором выполнялись замеры времени, представлены далее:

\begin{itemize}[itemindent=1.25em]
	\item[---] операционная система Windows 11 Pro Версия 22H2 (22621.674) \cite{wind};
	\item[---] память 16 ГБ;
	\item[---] процессор 11th Gen Intel(R) Core(TM) i5-11400 2.59 ГГц \cite{proc}, 6 физических и 12 логических ядер.
\end{itemize}

При тестировании компьютер был включен в сеть электропитания. Во время замеров процессорного времени устройство было нагружено только встроенными приложениями окружения, а также системой тестирования.

\section[Время выполнения реализаций алгоритмов]{Время выполнения реализаций\\алгоритмов}

На рис. \ref{img:graph1} и рис. \ref{img:graph} показаны результаты замеров времени.

\begin{center}
\begin{tikzpicture}
	\begin{axis}[
		axis lines = left,
		xlabel = {Количество вершин},
		ylabel = {Время (тики)},
		legend pos=north west,
		ymajorgrids=true
		]
		
		
		\addplot[color=blue] table[x index=0, y index= 1] {src/Brute.txt}; 
		\addlegendentry{Полный перебор}
		
		\addplot[color=orange] table[x index=0, y index= 1] {src/Ant.txt}; 
		\addlegendentry{Муравьиный алгоритм}
		
		
	\end{axis}
\end{tikzpicture}
		\captionof{figure}{Сравнение времени работы алгоритмов при увеличении размера графа}
\label{img:graph1}
\end{center}
\begin{center}

\begin{tikzpicture}
	\begin{axis}[
		axis lines = left,
		xlabel = {Количество вершин},
		ylabel = {Время (тики)},
		legend pos=north west,
		ymajorgrids=true
		]
		
		
		\addplot[color=blue] table[x index=0, y index= 1] {src/Brute2.txt}; 
		\addlegendentry{Полный перебор}
		
		\addplot[color=orange] table[x index=0, y index= 1] {src/Ant2.txt}; 
		\addlegendentry{Муравьиный алгоритм}
		
		
	\end{axis}

\end{tikzpicture}
		\captionof{figure}{Сравнение времени работы алгоритмов на малых размерах графа}
\label{img:graph}
\end{center}

\section{Параметризация муравьиного алгоритма}	

Для различных значений параметров $\alpha$, $\beta$, $\rho$ и $t_{max}$ для каждой из нескольких матриц смежности с помощью муравьиного алгоритма и перебора была найдена некоторая длина маршрута. Далее выбраны наилучшие сочетания параметров муравьиного алгоритма на этих данных.\\
Параметр $\alpha$ менялся от 0 до 10, параметр $\rho$ менялся от 0.1 до 0.9, параметр $t_{max}$ менялся от 5 до 100.\\
<<<<<<< HEAD
Итого были выявлены оптимальные сочетания параметров (представлены в таблице \ref{tbl:best}):\\
=======
Итого были выявлены оптимальные сочетания параметров (представлены в таблице 4.1):\\
>>>>>>> 33115874deba1850798fb4b2d715e9d23f6db5b3

\begin{center}
	\begin{threeparttable}
		\caption{Результаты решения задачи параметризации}
		\label{tbl:best}
<<<<<<< HEAD
		\begin{tabular}{|p{2cm}|p{2cm}|p{2cm}|p{2cm}|}
=======
		\begin{tabular}{|p{1cm}|p{1cm}|p{2cm}|p{2cm}|}
>>>>>>> 33115874deba1850798fb4b2d715e9d23f6db5b3
			\hline
			$\alpha$ &$\beta$ & $\rho$ & $t_{max}$ \\\hline
			4&6&0.6&20\\
			6&4&0.3&40\\
			1&9&0.7&50\\
			6&4&0.9&70\\
			3&7&0.6&80\\
			6&4&0.25&90\\
			\hline
		\end{tabular}
<<<<<<< HEAD
	\end{threeparttable}
\end{center} 


\section{Вывод}
По полученным результатам иследования можно сделать вывод, что на больших размерностях (размер графа больше 9) полный перебор крайне медленен относительно муравьиного алгоритма.
=======
		\end{threeparttable}
\end{center}


\section{Вывод}
По полученным результатам иследования можно сделать вывод, что после на больших размерностях (размер графа больше 9) полный перебор крайне медленен относительно муравьиного алгоритма.
>>>>>>> 33115874deba1850798fb4b2d715e9d23f6db5b3
