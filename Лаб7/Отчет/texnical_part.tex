\chapter{Технологическая часть}
В данном разделе будут рассмотрены средства реализации, а также представлены листинги алгоритмов.
\section[Требования к программному обеспечению]{Требования к программному\\обеспечению }

Программа должна предоставлять пользователю три функции:
\begin{enumerate}[itemindent=1em]
	\item[1)] поиск гамильтонова цикла с помощью алгоритма полного перебора;
	\item[2)] поиск гамильтонова цикла с помощью муравьиного алгоритма;
	\item[3)] параметризация муравьиного алгоритма.
\end{enumerate}

Притом для решения задачи коммивояжера используется файл с матрицей расстояний, имя которого пользователь должен ввести по запросу программы.

Программа должна обрабатывать ошибки (например, отсутствие файла с матрицей расстояний) и корректно завершать работу с выводом информации об ошибке на экран.

\section{Средства реализации}
В данной работе для реализации был выбран язык программирования $c\#$. В текущей лабораторной работе требуется замерить время работы реализаций алгоритмов. Для этого используется класс  $Stopwatch$ из библиотеки $System.Diagnostics$.
\newpage
\section{Сведения о модулях программы}
Программа состоит из следующих классов:
\begin{itemize}[itemindent=1.25em]
	\item[---] $AntAlgorithm$ --- класс, реализующий муравьиный алгоритм; 
	\item[---] $Program$ --- класс, реализующий основную программу;
    \item[---] $BruteForce$ --- класс, реализующий алгоритм полного перебора;
\end{itemize}

\section{Реализация алгоритмов}
В листингах \ref{l1}--\ref{l2} представлены реализации алгоритмов. 
\begin{center}
\begin{lstlisting}[label=l1, caption={Реалитзация алгоритма полного перебора}]
 class BruteForce
{
	public static Path GetShortestPath(Map m)
	{
		int[] routeIndexes = new int[m.n - 1];
		for (int i = 0; i < m.n - 1; i++)
		{
			routeIndexes[i] = i + 1;
		}
		var allRoutes = GetAllRoutes(routeIndexes); 
		Path shortestPath = new Path(int.MaxValue);
		foreach (List<int> path in allRoutes)
		{
			path.Insert(0, 0);
			path.Add(0);
			Path cur = new Path(m, path.ToArray());
			if (cur.distance < shortestPath.distance)
			shortestPath = cur;
		}
		
		return shortestPath;
	}
	
	public static List<List<T>> GetAllRoutes<T>(IList<T> arr, List<List<T>> res = null, List<T> current = null)
	{
		if (res == null)
		res = new List<List<T>>();
		if (arr.Count == 0) 
		{
			res.Add(current);
			return res;
		}
		for (int i = 0; i < arr.Count; i++)
		{
			List<T> lst = new List<T>(arr);
			lst.RemoveAt(i);
			List<T> next;
			if (current == null)
			next = new List<T>();
			else
			next = new List<T>(current);
			next.Add(arr[i]);
			GetAllRoutes(lst, res, next);
		}
		return res;
	}
}
\end{lstlisting}
\end{center}
\begin{center}
\begin{lstlisting}[label=l2, caption={Реализация муравьиного алгоритма}]
     public static Path GetShortestPath(Map map, int nIter, double alpha, double beta, double Q, double ro)
{
	Path minPath = new Path(int.MaxValue);
	int n = map.n;
	double[][] pheromones = Program.InitMatr(n, 0.1);
	List<Ant> ants = new List<Ant>();
	for (int t = 0; t < nIter; t++)
	{
		ants = InitAnts(map, n);
		for (int i = 0; i < n - 1; i++)
		{
			double[][] pheromonesIter = Program.InitMatr(n, (double)0); 
			for (int j = 0; j < ants.Count(); j++) 
			{
				double sumChance = 0, chance = 0;
				bool flag = false;
				Ant curAnt = ants[j];
				int curCity = curAnt.path.route.Last();
				for (int sityId = 0; sityId < n; sityId++) 
				{
					if (curAnt.visited[sityId] == false)
					{
						sumChance += Math.Pow(pheromones[curCity][sityId], alpha) * Math.Pow(1 / (map.distance[curCity,sityId]), beta); 
						flag = true;
					}
				}
				double x = r.NextDouble();
				int k = 0;
				for (; x > 0; k++)
				{
					if (curAnt.visited[k] == false)
					{
						chance = Math.Pow(pheromones[curCity][k], alpha) * Math.Pow(1 / (map.distance[curCity,k]), beta);
						chance /= sumChance;
						x -= chance;
					}
				}
				k--;
				ants[j].VisitedTown(k);
				pheromonesIter[curCity][k] += Q / (map.distance[curCity,k]);
			}
			for (int ii = 0; ii < n; ii++)
			for (int j = 0; j < n; j++)
			pheromones[ii][j] = (1 - ro) * pheromones[ii][j] + pheromonesIter[ii][j];
		}
		
		foreach (Ant a in ants)
		{
			a.VisitedTown(a.iStartTown);
			if (a.GetDistance() < minPath.distance)
			{
				minPath = a.path;
			}
		}                
	}
	return minPath;
}
public static List<Ant> InitAnts(Map m, int n)
{
	List<Ant> ants = new List<Ant>();
	for (int i = 0; i < n; i++)
	{
		ants.Add(new Ant(m, 0));
	}
	return ants;
}
\end{lstlisting}
\end{center}
\section{Функциональное тестирование}
В таблице \ref{tabular:testsdata} приведены тесты для функций, реализующих алгоритмы для решения задачи коммивояжера (поиска гамильтонова цикла). Тесты для всех реализаций алгоритмов пройдены успешно.

\begin{center}
\begin{threeparttable}
	\caption{Тестовые данные}
	\label{tabular:testsdata}
	\begin{tabular}{|p{6cm}|p{5cm}|p{4cm}|}
		\hline
		\textbf{Матрица расстояний} & \textbf{Алгоритм полного перебора} & \textbf{Муравьиный алгоритм}
		\tabularnewline
		\hline
		\begin{equation*}
			\begin{bmatrix}
				0
			\end{bmatrix}
		\end{equation*} & 0; [1, 1] & 0; [1, 1]
		\tabularnewline
		\hline
		\begin{equation*}
			\begin{bmatrix}
				0 & 10 \\
				10 & 0 \\
			\end{bmatrix}
		\end{equation*} & 10; [1, 2, 1] & 10; [1, 2, 1]
		\tabularnewline
		\hline
		\begin{equation*}
			\begin{bmatrix}
				0 & 10 & 15 & 20 \\
				10 & 0 & 35 & 25 \\
				15 & 35 & 0 & 30 \\
				20 & 25 & 30 & 0 \\
			\end{bmatrix}
		\end{equation*} & 80; [1, 2, 4, 3, 1] & 80; [1, 2, 4, 3, 1]
		\tabularnewline
		\hline
	\end{tabular}
\end{threeparttable}
\end{center}

\section*{\hsp Вывод}
% \addcontentsline{toc}{section}{Вывод}
В данном разделе были рассмотрены листинги используемых алгориитмов, проведено функциональное тестирование.
