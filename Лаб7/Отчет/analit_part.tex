\chapter{Аналитическая часть}
\section{Задача коммивояжера}

В задаче коммивояжера рассматривается $n$ городов и матрица попарно различных расстояний между ними.
Требуется найти такой порядок посещения городов, чтобы суммарное пройденное расстояние было минимальным, каждый город посещался ровно один раз.
Иногда условием задачи коммивояжера является возврат в тот город, с которого начинался маршрут.

\section{Алгоритм полного перебора}

Для решения задачи коммивояжера алгоритм полного перебора предполагает рассмотрение всех возможных путей в графе и выбор наименьшего из них. 
Смысл перебора состоит в том, что перебираются все варианты объезда городов и выбирается оптимальный, что гарантирует точное решение задачи.
Однако, при таком подходе количество возможных маршрутов очень быстро возрастает с ростом $n$ (сложность алгоритма равна $n!$).

\section{Муравьиный алгоритм}

Муравьиный алгоритм --- метод решения задач коммивояжера на основании моделирования поведения колонии муравьев.

Каждый муравей определяет для себя маршрут, который необходимо пройти на основе феромона, который он ощущает во время прохождения, каждый муравей оставляет феромон на своем пути, чтобы остальные муравьи могли по нему ориентироваться. В результате при прохождении каждым муравьем различного маршрута наибольшее число феромона остается на оптимальном пути.


Для каждого муравья переход из города i в город j зависит от трех составляющих: памяти муравья, видимости и виртуального следа феромона.

\begin{itemize}[itemindent=1.25em]
	\item[---] Память муравья --- это список посещенных муравьем городов, заходить в которые еще раз нельзя.
	Используя этот список, муравей гарантированно не попадет в один и тот же город дважды. 
	Данный список возрастает при совершении маршрута и обнуляется в начале каждой итерации алгоритма.
	
	\item[---] Видимость --- величина, обратная расстоянию, рассчитывающаяся по формуле \ref{d_func}.
	\begin{equation}
		\label{d_func}
		\eta_{ij} = 1 / D_{ij},
	\end{equation} 
	где $D_{ij}$ --- расстояние между городами $i$ и $j$. 
	Видимость --- это локальная статическая информация, выражающая эвристическое желание посетить город $j$ из города $i$, то есть чем ближе город, тем больше желание посетить его.
	
	\item[---] Виртуальный след феромона на ребре $(i, j)$ представляет подтвержденное муравьиным опытом желание посетить город $j$ из города $i$. Cлед феромона является глобальной и динамичной информацией --- она изменяется после каждой итерации алгоритма, отражая приобретенный муравьями опыт.
\end{itemize}

Формула вычисления вероятности перехода в заданную точку \eqref{posib}.

\begin{equation}
	\label{posib}
	P_{kij} = \begin{cases}
		\frac{\tau_{ij}^a\eta_{ij}^b}{\sum_{q=1}^m \tau^a_{iq}\eta^b_{iq}}, \textrm{вершина не была посещена ранее муравьем k,} \\
		0, \textrm{иначе,}
	\end{cases}
\end{equation}
где $a$ --- параметр влияния длины пути, $b$ --- параметр влияния феромона, $\tau_{ij}$~--- расстояния от города $i$ до $j$, $\eta_{ij}$ --- количество феромонов на ребре $(i, j)$.

Правило обновления феромона после движения всех муравьев:
\begin{equation}
	\label{update_phero_1}
	\tau_{ij}(t+1) = (1-p)\tau_{ij}(t) + \Delta \tau_{ij}.
\end{equation}

При этом
\begin{equation}
	\label{update_phero_2}
	\Delta \tau_{ij} = \sum_{k=1}^N \tau^k_{ij},
\end{equation}
где
\begin{equation}
	\label{update_phero_3}
	\Delta\tau^k_{ij} = \begin{cases}
		Q/L_{k}, \textrm{ребро посещено k-ым муравьем,} \\
		0, \textrm{иначе.}
	\end{cases}
\end{equation}

Существует несколько оптимизаций данного алгоритма, одна из которых введение так называемых элитных муравьев. Элитный муравей усиливает ребра наилучшего маршрута, найденного с начала работы алгоритма. Количество феромона, откладываемого на ребрах наилучшего текущего
маршрута $T^+$, принимается равным $Q/L^+$, где $L^+$ — длина маршрута $T^+$. Этот феромон побуждает муравьев к исследованию решений, содержащих несколько ребер наилучшего на данный момент маршрута $T^+$. Если в муравейнике есть $e$ элитных муравьев, то ребра маршрута $T^+$ будут получать общее усиление:

\begin{equation}
	\label{opt}
	\Delta \tau_{e} = e * Q/L^+.
\end{equation}

\section*{\hsp Вывод}

В этом разделе была изучена задача коммивояжера и используемые для нее решения алгоритмы: полный перебор и муравьиный алгоритм с оптимизацией в виде элитных муравьев.
