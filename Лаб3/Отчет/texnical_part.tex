\chapter{Технологическая часть}
В данном разделе будут рассмотрены средства реализации, а также представлены листинги сортировок.
\section{Средства реализации}
В данной работе для реализации был выбран язык программирования $c++$. В текущей лабораторной работе требуется замерить процессорное время для выполняемой программы, а также построить графики. Для визуализации результатов использовался язык $Python$.

Время работы было замерено с помощью функции \textit{GetProcessTimes(...)} \cite{time} из библиотеки $Windows.h$.

\section{Сведения о модулях программы}
Программа состоит из следующих модулей:
\begin{itemize}
	\item $ConsoleApplication1.cpp$ - файл, содержащий весь служебный код;
	\item $sorst.cpp$ - файл, содержащий код всех сортировок;
	\item $time\_compare.cpp$ - файл, производящий замеры времени;
	\item $getCPUTime.cpp$ - файл, определяющий функцию замера времени;
	\item $arr\_gen.cpp$ - файл, генерирующий входной массив;
	\item $sorst.hpp$ - заголовочный файл модуля $sorst.cpp$;
	\newpage
	\item $time\_compare.hpp$ - заголовочный файл модуля $time\_compare.cpp$;
	\item $getCPUTime.hpp$ - заголовочный файл модуля $getCPUTime.cpp$;
	\item $arr\_gen.hpp$ - заголовочный файл модуля $arr\_gen.cpp$.\newline
\end{itemize}

\section{Реализация алгоритмов}
В листингах \ref{merge_sort}, \ref{smooth_sort}, \ref{bucket_sort}, представлены реализации алгоритмов сортировок (слиянием, плавной, блочной). В листнингах \ref{merge}, \ref{findPosMaxElem}, \ref{make_heap_pool}, \ref{shiftDown}, \ref{NextState}, \ref{PrevState} представлены реализации вспомогательных функций. \clearpage
\begin{center}
\begin{lstlisting}[label=merge_sort, caption={Алгоритм сортировки слиянием}]
	void merge_sort(vector<int>& arr, int low, int high)
	{
		int mid;
		if (low < high) {
			mid = (low + high) / 2;
			merge_sort(arr, low, mid);
			merge_sort(arr, mid + 1, high);
			merge(arr, low, high, mid);
		}
	}
\end{lstlisting}
\end{center}
\newpage
\begin{center}
\begin{lstlisting}[label=merge, caption={Алгоритм функции merge}]
void merge(vector<int>& arr, int low, int high, int mid)
{
	int i, j, k, c[50];
	i = low;
	k = low;
	j = mid + 1;
	while (i <= mid && j <= high) {
		if (arr[i] < arr[j]) {
			c[k] = arr[i];
			k++;
			i++;
		}
		else {
			c[k] = arr[j];
			k++;
			j++;
		}
	}
	while (i <= mid) {
		c[k] = arr[i];
		k++;
		i++;
	}
	while (j <= high) {
		c[k] = arr[j];
		k++;
		j++;
	}
	for (i = low; i < k; i++) {
		arr[i] = c[i];
	}
}
\end{lstlisting}
\end{center}
\newpage
\begin{center}
\begin{lstlisting}[label=smooth_sort,caption = { Алгоритм плавной сортировки}]
typedef int INT;
	
L(N) = L(N-1) + L(N-2) + 1
int LeoNum[] = { 1, 1, 3, 5, 9, 15, 25, 41, 67, 109, 177, 287, 465, 753, 1219, 1973, 3193, 5167, 8361, 13529, 21891, 35421, 57313, 92735, 150049, 242785, 392835, 635621, 1028457, 1664079, 2692537, 4356617, 7049155, 11405773, 18454929, 29860703, 48315633, 78176337, 126491971, 204668309, 331160281, 535828591, 866988873, 1402817465 };
INT curState;

void smooth_sort(vector<int>& mas)
{
	make_heap_pool(mas);
	
	for (int i = mas.size() - 1; i >= 0; i--)
	{
		int nextPosHeapItemsAmount;
		int posMaxTopElem = findPosMaxElem(mas, curState, i, nextPosHeapItemsAmount);
		if (posMaxTopElem != i)
		{
			swap(mas[i], mas[posMaxTopElem]);
			shiftDown(mas, nextPosHeapItemsAmount, posMaxTopElem);
		}
		PrevState(curState);
	}
}
\end{lstlisting}
\end{center}
\newpage
\begin{center}
\begin{lstlisting}[label=findPosMaxElem, caption={Алгоритм функции findPosMaxElem}]
int findPosMaxElem(vector<int>& mas, INT curState, int indexLastTop, int& nextPosHeapItemsAmount)
{
	int pos = 0;
	while (!(curState & 1))
	{
		curState >>= 1;
		pos++;
	}
	
	int posMaxTopElem = indexLastTop;
	nextPosHeapItemsAmount = pos;
	
	int curTopElem = indexLastTop - LeoNum[pos];
	
	curState >>= 1;
	pos++;
	while (curState)
	{
		if (curState & 1)
		{
			if (mas[curTopElem] > mas[posMaxTopElem])
			{
				posMaxTopElem = curTopElem;
				nextPosHeapItemsAmount = pos;
			}
			curTopElem -= LeoNum[pos];
		}
		curState >>= 1;
		pos++;
	}
	
	return posMaxTopElem;
}
\end{lstlisting}
\end{center}
\newpage
\begin{center}
\begin{lstlisting}[label=make_heap_pool, caption={Алгоритм функции make\_heap\_pool}]
void make_heap_pool(vector<int>& mas)
{
	for (size_t i = 0; i < mas.size(); i++)
	{
		int posHeapItemsAmount = NextState(curState);
		if (posHeapItemsAmount != -1)
		shiftDown(mas, posHeapItemsAmount, i);
	}
}
\end{lstlisting}
\end{center}
\newpage
\begin{center}
\begin{lstlisting}[label=shiftDown, caption={Алгоритм функции shiftDown}]
void shiftDown(vector<int>& mas, int posHeapItemsAmount, int indexLastTop)
{
	int posCurNode = indexLastTop;
	while (posHeapItemsAmount > 1)
	{
		int posR = posCurNode - 1;
		int posL = posR - LeoNum[posHeapItemsAmount - 2];
		
		int posMaxChild = posL;
		int posNextTop = posHeapItemsAmount - 1;
		if (mas[posR] > mas[posL])
		{
			posMaxChild = posR;
			posNextTop = posHeapItemsAmount - 2;
		}
		if (mas[posCurNode] < mas[posMaxChild])
		{
			swap(mas[posCurNode], mas[posMaxChild]);
			posHeapItemsAmount = posNextTop;
			posCurNode = posMaxChild;
		}
		else
		break;
	}
}
\end{lstlisting}
\end{center}
\newpage
\begin{center}
\begin{lstlisting}[label=PrevState, caption={Алгоритм функции PrevState}]
void PrevState(INT& curState)
{
	if ((curState & 15) == 8)
	{             
		curState -= 3; 
	}
	else
	if (curState & 1)
	{
		if ((curState & 3) == 3) 
		curState ^= 2;    
		else            
		curState ^= 1;      
	}
	else
	{
		INT prev = curState;
		int pos = 0;
		while (prev && !(prev & 1))
		{
			prev >>= 1;
			pos++;
		}
		if (prev)                  
		{
			curState ^= 1 << pos;
			curState |= 1 << (pos - 1);
			curState |= 1 << (pos - 2);
		}
		else
		curState = 0;
	}
}
\end{lstlisting}
\end{center}
\newpage
\begin{center}
\begin{lstlisting}[label=NextState, caption={Алгоритм функции NextState}]
int NextState(INT& curState)
{
	int posNewTop = -1;

	if ((curState & 7) == 5)
	{                   
		curState += 3;  
		posNewTop = 3;
	}
	else  
	{
		INT next = curState;
		int pos = 0;
		while (next && (next & 3) != 3)
		{
			next >>= 1;
			pos++;
		}
		if ((next & 3) == 3)   
		{
			curState += 1 << pos; 
			posNewTop = pos + 2;
		}
		else
		if (curState & 1)   
		curState |= 2;     
		else                    
		curState |= 1;    
	}
	return posNewTop;
}
\end{lstlisting}
\end{center}
\newpage
\begin{center}
\begin{lstlisting}[label=bucket_sort, caption={Алгоритм блочной сортировки}]
void bucketSort(vector<int>& arr, int n, int number_of_buckets)
{
	vector<vector<int>> b;
	int max = arr[0], min = arr[0];
	
	for (int i = 0; i < n; i++)
	{
		if (arr[i] < min)
		min = arr[i];
		if (arr[i] > max)
		max = arr[i];
	}
	
	for (int i = 0; i < number_of_buckets; i++)
	{
		vector<int> k;
		b.push_back(k);
	}
	
	for (int i = 0; i < n; i++)
	{
		int x = floor(number_of_buckets * (arr[i] - min) / (max + 1 - min));
		b[x].push_back(arr[i]);
	}
	for (int i = 0; i < number_of_buckets; i++)
	smooth_sort(b[i]);
	int index = 0;
	for (int i = 0; i < number_of_buckets; i++)
	for (int j = 0; j < b[i].size(); j++)
	arr[index++] = b[i][j];
}
\end{lstlisting}
\end{center}
\newpage
\section{Функциональные тесты}

В таблице \ref{tbl:functional_test} приведены тесты для функций, реализующих алгоритмы сортировки. Тесты для всех сортировок пройдены успешно.



	\begin{center}
		\begin{threeparttable}
			
				\caption{\label{tbl:functional_test} Функциональные тесты}
			\begin{tabular}{|c|c|c|}
				\hline
				Входной массив & Ожидаемый результат & Результат \\ 
				\hline
				$[1, 2, 3, 4, 5]$ & $[1, 2, 3, 4, 5]$  & $[1, 2, 3, 4, 5]$\\
				$[9, 7, 5, 1, 4]$  & $[1, 4, 5, 7, 9]$  & $[1, 4, 5, 7, 9]$\\
				$[5, 4, 3, 2, 1]$  & $[1, 2, 3, 4, 5]$ & $[1, 2, 3, 4, 5]$\\
				$[5]$  & $[5]$  & $[5]$\\
				$[]$  & $[]$  & $[]$\\
				\hline
			\end{tabular}
	
		\end{threeparttable}
	\end{center}



\section*{Вывод}
% \addcontentsline{toc}{section}{Вывод}
В данном разделе были представлены реализации алгоритмов плавной сортировки, сортировки слиянием и блочной сортировки.