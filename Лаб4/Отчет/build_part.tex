\chapter{Конструкторская часть}

\section{Описание алгоритма}
Рассмотрим подробнее каждый шаг.

На первом шаге устанавливаются размеры холста, расстояние от наблюдателя до холста и местоположение наблюдателя в трехмернном пространстве.

Второй шаг не представляет реальной сложности из-за условий расположения холста. Для перехода от координат рамки к координатам пространства необходимо изменить масштаб. Пусть $C_{x}, C_{y}$ координаты пикселя на холсте по осям $X, Y$ соответственно. Тогда координаты точки в пространстве будут вычислятся по следующей формуле, где $V_{w}, V_{h}$ и $C_{w}, C_{h}$ размеры по ширине и высоте для пространства и холста соответсвенно: 
\[V_{x} = C_{x} * \frac{V_{w}}{C_{w}}\]
\[V_{y} = C_{y} * \frac{V_{h}}{C_{h}}\]

Третий шаг самый сложный и трудоемкий в данном алгоритме. Для определения цвета необхлдимо определить какая фигура находится ближе всего к наблюдателю в данной точке. Для этого используются параметрические уравнения. Для луча, исходящего от наблюдателя оно задается следующим образом: 
\[P = O  + t(V - O)\]
где $O$ --- положение наблюдателя, $V$ --- положение точки в пространстве, $t$ --- произвольное действительное число. Для оптимизации алгоритма принято простейщие примитивы описывать в сферы. Уравнение сферы для $P$ --- точки на сфере, $C$ --- центра сферы и $r$ --- радиуса сферы принимает вид:\newpage \[<P - C, P - C> = r^{2}\]
Тогда для нахождения точки пересечения необходимо решить квадратное уравнение, где $\vec{D} = V - O$
\[t^{2}<\vec{D},\vec{D}> + 2t<\vec{D},\vec{OC}> + <\vec{OC},\vec{OC}> - r^{2} = 0\]
В случае, если объектом является не сфера, то вычисления становятся более трудоемкими и более затратными по времени.

\section{Параллелизация алгоритма}
В связи с тем, что данный алгоритм проходит по всем пикселям холста, то с увеличением размеров холста увеличивается и время работы алгоритма. 

Разные части изображения могут обрабатываться независимо от других. По этой причине можно обрабатывать части холста одновременно, используя потоки, таким образом уменьшая общее время работы реализации алгоритма.

Разделим холст по оси $X$ на количество логических ядер процессора. Именно такое количество потоков будет оптимально. Для передачи данных в поток создадим структуру, в которой будут содержаться начало и конец текущего отрезка для обработки.

%\section*{Вывод}
