\chapter{Технологическая часть}
В данном разделе будут рассмотрены средства реализации, а также представлены листинги сортировок.
\section{Средства реализации}
В данной работе для реализации был выбран язык программирования $c\#$. В текущей лабораторной работе требуется замерить время выполнения реализаци трассировщика лучей.

Время работы было замерено с помощью класса \textit{Stopwatch} из библиотеки $System.Diagnostics$ \cite{time}. 

Для создания потоков был выбран класс \textit{Tread} из библиотеки $System.Threading$.

\section{Сведения о модулях программы}
Программа состоит из следующих модулей:
\begin{itemize}
	\item $Scene.cs$ - файл, содержащий сцену;
	\item $Form1.cs$ - файл, реализующий интерфейс программы;
	\item $RayTraiser.cs$ - файл, содержащий трассировку лучей.
\end{itemize}

\section{Реализация алгоритмов}
В данной работе используется параметризированный поток, который на вход \newpage принимает структуру, содержащую начало, конец интервала для трассировки и остальные параметры, требуемые для работы алгоритма.

В листинге \ref{common_part} представлена реализация общей части: функция для расчета пересечения луча и объекта, функция определяющая ближайший объект к наблюдателю, функция построения выпуклой оболочки -- для трассировки лучей. В листингах \ref{parallel_traise} и \ref{traise} представлены реализации параллельной и не паралелльной трассировки лучей соответсвенно. В листинге \ref{struct} представлена реализация структуры для передачи в поток.\clearpage
\begin{center}
\begin{lstlisting}[label=common_part, caption={Реализация общей части трассировки лучей}]
   private Color RayT(Model model, MatrixCoord3D D, MatrixCoord3D position)
{
	
	PolygonComponent closest = null;
	double closest_t = double.MaxValue;
	//  Parallel.ForEach<PolygonComponent>(model.Polygons, p=> { 
		foreach (PolygonComponent p in model.Polygons)
		{
			//   if (MatrixCoord3D.scalar(p.Normal, cam.Direction) > 0)
			{
				MatrixCoord3D tt = GetTimeAndUvCoord(position, D, p.Points[0].Coords, p.Points[1].Coords, p.Points[2].Coords);
				if (tt != null)
				{
					if (tt.X < closest_t && tt.X > 1)
					{
						closest_t = tt.X;
						closest = p;
					}
				}
			}
		}
		
		if (closest == null)
		return Color.White;
		return closest.ColorF;
	}
	private Tuple<MatrixCoord3D, double> FoundCenter(Container<PointComponent> points)
	{
		double minX, maxX, minY, maxY, minZ, maxZ;
		minX = points[0].X;
		maxX = points[0].X;
		minY = points[0].Y;
		maxY = points[0].Y;
		minZ = points[0].Z;
		maxZ = points[0].Z;
		
		foreach (PointComponent p in points)
		{
			if (p.X < minX)
			minX = p.X;
			if (p.X > maxX)
			maxX = p.X;
			if (p.Y < minY)
			minY = p.Y;
			if (p.Y > maxY)
			maxY = p.Y;
			if (p.Z < minZ)
			minZ = p.Z;
			if (p.Z > maxZ)
			maxZ = p.Z;
		}
		MatrixCoord3D position = new MatrixCoord3D((minX + maxX) / 2f, (minY + maxY) / 2f, (minZ + maxZ) / 2f);
		double r = 0;
		foreach (PointComponent p in points)
		{
			LineComponent minx = new LineComponent(new PointComponent(position), p);
			r = Math.Max(minx.Len(), r);
		}
		
		return new Tuple<MatrixCoord3D, double>(position, r);
	}
	private double RaySphereIntersection(MatrixCoord3D rayOrigin, MatrixCoord3D rayDirection, MatrixCoord3D spos, double r)
	{
		double t = Double.MaxValue;
		//a == 1; // because rdir must be normalized
		MatrixCoord3D k = rayOrigin - spos;
		double b = MatrixCoord3D.scalar(k, rayDirection);
		double c = MatrixCoord3D.scalar(k, k) - r * r;
		double d = b * b - c;
		if (d >= 0)
		{
			double sqrtfd = Math.Sqrt(d);
			// t, a == 1
			double t1 = -b + sqrtfd;
			double t2 = -b - sqrtfd;
			double min_t = Math.Min(t1, t2);
			double max_t = Math.Max(t1, t2);
			t = (min_t >= 0) ? min_t : max_t;
		}
		return t;
	}
	
	private const double Epsilon = 0.000001d;
	
	private MatrixCoord3D? GetTimeAndUvCoord(MatrixCoord3D rayOrigin, MatrixCoord3D rayDirection, MatrixCoord3D vert0, MatrixCoord3D vert1, MatrixCoord3D vert2)
	{
		var edge1 = vert1 - vert0;
		var edge2 = vert2 - vert0;
		
		var pvec = (rayDirection * edge2);
		
		var det = MatrixCoord3D.scalar(edge1, pvec);
		
		if (det > -Epsilon && det < Epsilon)
		{
			return null;
		}
		
		var invDet = 1d / det;
		
		var tvec = rayOrigin - vert0;
		
		var u = MatrixCoord3D.scalar(tvec, pvec) * invDet;
		
		if (u < 0 || u > 1)
		{
			return null;
		}
		
		var qvec = (tvec * edge1);
		
		var v = MatrixCoord3D.scalar(rayDirection, qvec) * invDet;
		
		if (v < 0 || u + v > 1)
		{
			return null;
		}
		
		var t = MatrixCoord3D.scalar(edge2, qvec) * invDet;
		
		return new MatrixCoord3D(t, u, v);
	}
	
	private MatrixCoord3D GetTrilinearCoordinateOfTheHit(float t, MatrixCoord3D rayOrigin, MatrixCoord3D rayDirection)
	{
		return rayDirection * t + rayOrigin;
	}
	
	private MatrixCoord3D CanvasToVieport(int x, int y, double aspect, double fov)
	{
		
		double fx = aspect * fov * (2 * ((x + 0.5f) / screen.Width) - 1);
		double fy = (1 - (2 * (y + 0.5f) / screen.Height)) * fov;
		
		return new MatrixCoord3D(fx / scale, fy / scale, -1);
	}
\end{lstlisting}
\end{center}
\newpage
\begin{center}
\begin{lstlisting}[label=parallel_traise, caption={Алгоритм паралельной трассировки}]
        public override void RayTrasing(Model model)
{
	
	Tuple<MatrixCoord3D, double> sphere = FoundCenter(model.Points);
	List<Thread> threads = new List<Thread>();
	
	MatrixCoord3D CamPosition = cam.Position.Coords;
	MatrixTransformation3D RotateMatrix = cam.RotateMatrix;
	
	int x = 0;
	for (int h = 0; h < NumberofThreads; h++)
	{
		threads.Add(new Thread(new ParameterizedThreadStart(ByVertecal)));
		threads[threads.Count - 1].Start(new Limit(x, x + screen.Width / NumberofThreads,model,CamPosition, RotateMatrix,sphere ));
		x += screen.Width / NumberofThreads;
	}
	foreach (var elem in threads)
	{
		elem.Join();
	}
}

private void ByVertecal(object obj)
{
	Limit limit = (Limit)obj;
	Color c;
	double aspect = screen.Width / screen.Height;
	double field = Math.Tan(cam.Fovy / 2 * Math.PI / 180.0f);
	for (int x = limit.begin; x < limit.end; x++)
	for (int y = 0; y < screen.Height; y++)
	{
		MatrixCoord3D D = CanvasToVieport(x, y,aspect,field) * limit.RotateMatrix;// new MatrixCoord3D(cam.RotateMatrix.Coeff[0,2], cam.RotateMatrix.Coeff[1, 2], cam.RotateMatrix.Coeff[2, 2]);
		D.Normalise();
		c = Color.White;
		if (RaySphereIntersection(limit.CamPosition, D, limit.sphere.Item1, limit.sphere.Item2) != double.MaxValue)
		{
			c = RayT(limit.model, D, limit.CamPosition);
		}
		PictureBuff.SetPixel(x, y, c.ToArgb());
	}
}
\end{lstlisting}
\end{center}
\newpage
\begin{center}
\begin{lstlisting}[label=traise,caption = { Алгоритм последовательной трассировки}]
        public virtual void RayTrasing(Model model)
{
	Tuple<MatrixCoord3D, double> sphere = FoundCenter(model.Points);
	MatrixCoord3D CamPosition = cam.Position.Coords;
	MatrixTransformation3D RotateMatrix = cam.RotateMatrix.InversedMatrix();
	double aspect = screen.Width / (double)screen.Height;
	double field = Math.Tan(cam.Fovy / 2 * Math.PI / 180.0f);
	for (int x = 0; x < screen.Width; x++)
	{
		for (int y = 0; y < screen.Height; y++)
		{
			MatrixCoord3D D = CanvasToVieport(x, y, aspect, field) * RotateMatrix;// new MatrixCoord3D(cam.RotateMatrix.Coeff[0,2], cam.RotateMatrix.Coeff[1, 2], cam.RotateMatrix.Coeff[2, 2]);
			D.Normalise();
			Color c = Color.White;
			if (RaySphereIntersection(CamPosition, D, sphere.Item1, sphere.Item2) != double.MaxValue)
			{
				c = RayT(model, D, CamPosition);
			}
			PictureBuff.SetPixel(x, y, c.ToArgb());
		}
	}
}

\end{lstlisting}
\end{center}
\newpage
\begin{center}
	\begin{lstlisting}[label=struct,caption = { Структура для передачи в поток}]
	  class Limit
	{
		public int begin;
		public int end;
		readonly public Model model;
		readonly public MatrixCoord3D CamPosition;
		readonly public MatrixTransformation3D RotateMatrix;
		readonly public Tuple<MatrixCoord3D, double> sphere;
		public Limit(int begin, int end, Model model, MatrixCoord3D CamPosition, MatrixTransformation3D RotateMatrix, Tuple<MatrixCoord3D, double> sphere)
		{
			this.begin = begin; this.end = end; this.model = model; 
			this.CamPosition = CamPosition; this.RotateMatrix = RotateMatrix.InversedMatrix();
			this.sphere = sphere;
		}
	}
		
	\end{lstlisting}
\end{center}
\newpage


%\section*{Вывод}
%В данном разделе были представлены реализации алгоритмов плавной сортировки, сортировки слиянием и блочной сортировки.