
\chapter*{Введение}
\addcontentsline{toc}{chapter}{Введение}
В современном мире большое внимание уделяется различным алгоритмам преобразования трехмерных объектов в двумерное пространство экрана. Их принято разделять на две ветви.
 
\begin{enumerate}
	\item Алгоритмы растеризации.
	\item Алгоритм трассировки лучей.
\end{enumerate}

В данной работе будет использован алгоритм трассировки лучей.

Целью данной лабораторной работы является анализ времени работы реализации алгоритма параллельной трассировки лучей и последовательной.

Для достижения поставленной цели требуется решить ряд задач:
\begin{enumerate}
	\item[1)] разработать последовательный алгоритм визуализации и вывода на экран заданной трехмерной модели с использованием трассировки лучей;
	\item[2)] разработать параллельную версию алгоритма;
	\item[3)] реализовать параллельный и последовательный алгоритмы трассировки лучей;
	\item[4)] разработать программное обеспечение для вывода на экран заданной трехмерной модели с использованием трассировки лучей;
	\item [5)] провести замеры времени для каждой из реализаций алгоритма;
	\item[6)] выполнить анализ полученных результатов;
	\item [7)] по итогам работы составить отчет.
\end{enumerate}
\newpage
