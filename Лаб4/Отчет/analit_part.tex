\chapter{Аналитическая часть}

\section{Описание алгоритма трассировки лучей}
Первые упоминания о трассировке лучей относятся к шестидесятым годам двадцатого века. Однако широкое распространение данный метод рендера получил относительно недавно. Связанно это с ростом производительности графических процессоров.

Алгоритм трассировки лучей работает идентично художнику рисующему картину на холсте. В связи с этим определим основные положения: наблюдатель --- точка, из которой проводится наблюдение, может свободно перемещаться по сцене; холст --- бесконечная плоскость перпендикулярная взгляду наблюдателя.

Из глаза наблюдателя выпускается луч, который проходит по каждой точке холста.

В данном алгоритме принято выделять четыре основных шага. Шаги 3--4 повторяются для каждого пикселя из холста.
\begin{enumerate}
	\item Поместить наблюдателя и рамку в необходимое место. 
	\item Определить квадрат сетки, соответствующий текущему пикселю.
	\item Определить цвет, видимый через этот квадрат.
	\item Закрасить пиксель этим цветом.
\end{enumerate}

