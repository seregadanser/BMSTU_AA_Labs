\chapter{Аналитическая часть}
\section{Матрица}
Пусть есть два конечных множества:
\begin{itemize}
	\item  номера строк $M = {1, 2, ..., m}$,
	\item  номера столбцов $N = {1, 2, ..., n}$, 
\end{itemize}
где $m$ и $n$ --- натуральные числа.
Тогда матрицей $A$  размера $m$ на $n$ называется структура вида:  
\begin{equation}
	\begin{pmatrix}
		a_{11} & a_{12} & \ldots & a_{1n}\\
		a_{21} & a_{22} & \ldots & a_{2n}\\
		\vdots & \vdots & \ddots & \vdots\\
		a_{m1} & a_{m2} & \ldots & a_{mn}
	\end{pmatrix};
\end{equation}
где элемент матрицы $a_{ij}$  находится на пересечении i-й строки и j-го столбца. При этом количество элементов матрицы равно $m * n$.

Можно выделить следующие операции над матрицами:
\begin{enumerate}
	\item[1)]  сложение матриц одинакового размера;
	\item[2)]  вычитание матриц одинакового размера;
	\item[3)]  умножение матриц в случае, если количество столбцов первой матрицы равно количеству строк второй матрицы. В итоговой матрице количество строк будет, как у первой матрицы, а столбцов -- как у второй. \newline
\end{enumerate}


\section{Описание алгоритмов}
В этом разделе будут рассмотрены следующие алгоритмы матричного умножения: классический, Винограда, оптимизированный Винограда.

\subsection{Классический алгоритм умножения матриц}

Пусть даны две матрицы

\begin{equation}
	A_{lm} = \begin{pmatrix}
		a_{11} & a_{12} & \ldots & a_{1m}\\
		a_{21} & a_{22} & \ldots & a_{2m}\\
		\vdots & \vdots & \ddots & \vdots\\
		a_{l1} & a_{l2} & \ldots & a_{lm}
	\end{pmatrix},
	\quad
	B_{mn} = \begin{pmatrix}
		b_{11} & b_{12} & \ldots & b_{1n}\\
		b_{21} & b_{22} & \ldots & b_{2n}\\
		\vdots & \vdots & \ddots & \vdots\\
		b_{m1} & b_{m2} & \ldots & b_{mn}
	\end{pmatrix},
\end{equation}
тогда матрица $C$
\begin{equation}
	C_{ln} = \begin{pmatrix}
		c_{11} & c_{12} & \ldots & c_{1n}\\
		c_{21} & c_{22} & \ldots & c_{2n}\\
		\vdots & \vdots & \ddots & \vdots\\
		c_{l1} & c_{l2} & \ldots & c_{ln}
	\end{pmatrix},
\end{equation}
где
\begin{equation}
	\label{eq:M}
	c_{ij} =
	\sum_{r=1}^{m} a_{ir}b_{rj} \quad (i=\overline{1,l}; j=\overline{1,n})
\end{equation}
будет называться произведением матриц $A$ и $B$.

Стандартный алгоритм  реализует данную формулу.


\subsection{Алгоритм Винограда}

Алгоритм Винограда --- алгоритм умножения матриц. Начальная версия имела асимптотическую сложность алгоритма примерно $O(n^{2,3755})$, где $n$ -- размер стороны матрицы, но после доработки он стал обладать лучшей асимптотикой среди всех алгоритмов умножения матриц \cite{CoppersmithWinograd}.

Рассмотрим два вектора $U = (u_1, u_2, u_3, u_4)$ и $W = (w_1, w_2, w_3, w_4)$.
Их скалярное произведение равно: $U \cdot W = u_1w_1 + u_2w_2 + u_3w_3 + u_4w_4$, что эквивалентно: % (\ref{for:new}):
\newpage
\begin{equation}
	\label{for:new}
	V \cdot W = (u_1 + w_2)(u_2 + w_1) + (u_3 + w_4)(u_4 + w_3) - u_1u_2 - u_3u_4 - w_1w_2 - w_3w_4.
\end{equation}

За счёт предварительной обработки данных можно получить прирост производительности: несмотря на то, что  полученное выражение требует большего количества операций, чем стандартное умножение матриц, выражение в правой части равенства можно вычислить заранее и запомнить для каждой строки первой матрицы и каждого столбца второй матрицы. 
Это позволит выполнить лишь два умножения и пять сложений, при учёте, что потом будет сложено только с двумя предварительно посчитанными суммами соседних элементов текущих строк и столбцов. 
Операция сложения выполняется быстрее, поэтому алгоритм должен работать быстрее обычного алгоритма перемножения матриц.

Стоит упомянуть, что при нечётном значении размера матрицы $M$ нужно дополнительно добавить к скалярному произведению векторов произведения крайних элементов соответствующих строк и столбцов.

\subsection{Оптимизация алгоритма Винограда}

При реализации рассмотренного выше алгоритма Винограда можно провести оптимизации.
\begin{enumerate}
	\item Операции сложения и вычитания с присваиванием следует реализовывать при помощи соответствующего оператора $+=$ или $-=$ (при наличии данных операторов в выбранном языке программирования).
	\item Операцию умножения на 2 программно эффективнее реализовывать как побитовый сдвиг влево на 1.
	\item Занесение в циклах вычисления множителей вычисления первых двух
	элементов во внутренний цикл j.
\end{enumerate}