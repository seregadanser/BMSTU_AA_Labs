\chapter{Конструкторская часть}

\section{Описание алгоритмов}
В данном разделе будут рассмотрены схемы алгоритмов классического уможения матриц (рис. \ref{ris:imageSS}), Винограда (рис. \ref{ris:imageSSO1}, \ref{ris:imageSSO2}, \ref{ris:imageSSO3}), оптимизированного Винограда (рис. \ref{ris:imageSS1}, \ref{ris:imageSS2}, \ref{ris:imageSS3}).
\begin{center}
	

\newpage
%\begin{figure}[h]
	\centering
	%\def\svgwidth{5cm} % используем для изменения размера, если надо
	\def\svgwidth{7cm}
	\input{src/standart.drawio.pdf_tex}
	\captionof{figure}{Схема стандартного алгоритма}
	\label{ris:imageSS}
%\end{figure}
\newpage
%\begin{figure}[h]
\centering
\def\svgwidth{3.5cm}
%\def\svgwidth{5cm} % используем для изменения размера, если надо
\input{src/vinoNO1.pdf_tex}
\captionof{figure}{Схема алгоритма Винограда, часть 1}
\label{ris:imageSSO1}
%\end{figure}
\newpage
%\begin{figure}[h]
\centering
\def\svgwidth{7cm}
%\def\svgwidth{5cm} % используем для изменения размера, если надо
\input{src/vinoNO2.pdf_tex}
\captionof{figure}{Схема алгоритма Винограда, часть 2}
\label{ris:imageSSO2}
%\end{figure}
\newpage
%\begin{figure}[h]
\centering
\def\svgwidth{12cm}
%\def\svgwidth{5cm} % используем для изменения размера, если надо
\input{src/vinoNO3.pdf_tex}
\captionof{figure}{Схема алгоритма Винограда, часть 3}
\label{ris:imageSSO3}
%\end{figure}
\newpage
%\begin{figure}[h]
	\centering
	\def\svgwidth{3.5cm}
	%\def\svgwidth{5cm} % используем для изменения размера, если надо
	\input{src/vino3.pdf_tex}
	\captionof{figure}{Схема оптимизированного алгоритма Винограда, часть 1}
	\label{ris:imageSS1}
%\end{figure}
\newpage
%\begin{figure}[h]
	\centering
	\def\svgwidth{8cm}
	%\def\svgwidth{5cm} % используем для изменения размера, если надо
	\input{src/vino2.pdf_tex}
	\captionof{figure}{Схема оптимизированного алгоритма Винограда, часть 2}
	\label{ris:imageSS2}
%\end{figure}
\newpage
%\begin{figure}[h]
	\centering
		\def\svgwidth{12cm}
	%\def\svgwidth{5cm} % используем для изменения размера, если надо
	\input{src/vino1.pdf_tex}
	\captionof{figure}{Схема оптимизированного алгоритма Винограда, часть 3}
	\label{ris:imageSS3}
%\end{figure}

\end{center}
\newpage

\section{Трудоемкость алгоритмов}
Введем модель вычисления трудоемкости.
\begin{enumerate}
	\item Операции из списка (\ref{for:operations}) имеют трудоёмкость, равную 1:
	\begin{equation}		
		+, -, /, *, \%, =, +=, -=, *=, /=, \%=, ==
		\nonumber
	\end{equation}
	\begin{equation}
		\label{for:operations}
		!=, <, >, <=, >=, [], ++, {-}-
	\end{equation}
	\item Трудоёмкость оператора выбора \code{if условие then A else B} рассчитывается как %(\ref{for:if})
	\begin{equation}
		\label{for:if}
		f_{if} = f_{\text{условия}} +
		\begin{cases}
			f_A, & \text{если условие выполняется,}\\
			f_B, & \text{иначе.}
		\end{cases}
	\end{equation}
	\item Трудоёмкость цикла из NN шагов рассчитывается как %(\ref{for:cycle}):
	\begin{equation}
		\label{for:cycle}
		f_{for} = f_{\text{инициализации}} + f_{\text{сравнения}} + NN(f_{\text{тела}} + f_{\text{инкремент}} + f_{\text{сравнения}})
	\end{equation}
	\item Трудоёмкость вызова функции равна 0.
\end{enumerate}




Далее будет приведены оценки трудоемкости алгоритмов. 
\subsection{Стандартный алгоритм умножения матриц}

Для стандартного алгоритма умножения матриц трудоемкость будет слагаться из трех пунктов:
\newpage
\begin{itemize}
	\item внешнего цикла по $i \in [1..M]$, трудоёмкость которого $f = 2 + M \cdot (2 + f_{body_{j}})$;
	\item цикла по $j \in [1..N]$, трудоёмкость которого $f = 2 +  N \cdot (2 + 2 + f_{body_{k}})$;
	\item цикла по $k \in [1..K]$, трудоёмкость которого $f = 2 + 13K$.
\end{itemize}

Поскольку трудоемкость стандартного алгоритма равна трудоемкости внешнего цикла, то
\begin{align}
	\label{for:standard}
	f_{standard} &= 2 + M \cdot (4 + N \cdot (4 + 13K)) = 2 + 4M + 4MN + 13MNK\\ &\approx 13MNK
\end{align}


\subsection{Алгоритм Винограда}

Чтобы вычислить трудоемкость алгоритма Винограда, нужно учесть следующие пункты.

\begin{itemize}
	\item Создание и инициализация массивов $a_{tmp}$ и $b_{tmp}$ трудоёмкостью %(\ref{for:init}):
	\begin{equation}
		\label{for:init}
		f_{init} = L + N
	\end{equation}
	
	\item Заполнение массива $a_{tmp}$, трудоёмкость: %(\ref{for:ATMP}):
	\begin{equation}
		\label{for:ATMP}
		f_{a_{tmp}} = 2 + L (4 + 7 +  (\frac{M}{2} - 1)  \cdot 12)
	\end{equation}
	
	\item Заполнение массива $b_{tmp}$, трудоёмкость: %(\ref{for:BTMP}):
	\begin{equation}
		\label{for:BTMP}
		f_{b_{tmp}} = 2 + N (4 + 7 + (\frac{M}{2} - 1) \cdot 12)
	\end{equation}
	
	\item Цикл заполнения для чётных размеров, трудоёмкость которого %(\ref{for:cycle}):
	\begin{equation}
		\label{sfor:cycle}
		f_{cycle} = 2 + L \cdot (4 + N \cdot (11 + \frac{M}{2} \cdot 23))
	\end{equation}
	
	\item Цикл, который необходим для рассчёта при нечётном размере матрицы, трудоемкость которого
	\newpage
	\begin{equation}
		\label{for:last}
		f_{last} = \begin{cases}
			2, & \text{чётная,}\\
			4 + L \cdot (4 + 14N), & \text{иначе.}
		\end{cases}
	\end{equation}
\end{itemize}

Тогда для худшего случая (нечётный общий размер матриц) имеем трудоемкость:
\begin{equation}
	\label{for:bad}
	f_{worst} =  f_{a_{tmp}} + f_{b_{tmp}} + f_{cycle} + f_{last}\approx 11.5 \cdot LNM
\end{equation}

Для лучшего случая (чётный общий размер матриц) ---
\begin{equation}
	\label{for:good}
	f_{best} =  f_{a_{tmp}} + f_{a_{tmp}} + f_{cycle} + f_{last} \approx 11.5 \cdot LNM
\end{equation}


\subsection{Оптимизированный алгоритм Винограда}

Оптимизация заключается в следующих пунктах.
\begin{itemize}
	\item Использование побитового сдвига вместо умножения на 2.
	\item Замена операции сложения и вычитания на операции $+=$ и $-=$ соответственно.
	\item Занесение в циклах вычисления множителей вычисления первых двух
	элементов во внутренний цикл j.
\end{itemize}

Тогда трудоемкость оптимизированного алгоритма Винограда состоит из нижележащих пунктов.

\begin{enumerate}
	\item Создание и инициализацие массивов $a_{tmp}$ и $b_{tmp}$ (\ref{for:init}).
	
	\item Заполнение массива $a_{tmp}$, трудоёмкость:
	\begin{equation}
		\label{for:ATMP1}
		f_{a_{tmp}} = 2 + L (4 + \frac{M}{2} \cdot 12)
	\end{equation}
	
	\item Заполнение массива $b_{tmp}$, трудоёмкость:
	\begin{equation}
		\label{for:BTMP1}
		f_{b_{tmp}} = 2 + N (4 + \frac{M}{2} \cdot 12)
	\end{equation}
	\newline
	\item Цикла заполнения для чётных размеров, трудоёмкость которого (\ref{for:cycle}).
	
	\item Условие, которое нужно для дополнительных вычислений при нечётном размере матрицы, трудоемкость которого (\ref{for:last}).
	
\end{enumerate}

Тогда для худшего случая --- при нечётной размерности матриц $M$ --- трудоемкость равна
\begin{equation}
	\label{for:bad_impr}
	f_{worst} = f_{MH} + f_{MV} + f_{cycle} + f_{last} \approx 11.5 \cdot LNM
\end{equation}

Для лучшего случая --- при чётной размерности матриц $M$ --- трудоемкость равна 
\begin{equation}
	\label{for:good_impr}
	f_{best} = f_{MH} + f_{MV} + f_{cycle} + f_{last} \approx 11.5 \cdot LNM
\end{equation}


\section*{Вывод}
% \addcontentsline{toc}{section}{Вывод}
В данном разделе были описаны и проанализированы три алгоритма умножения матриц. Рассчитаны трудоёмкости алгоритмов в лучшем случае (л.с.) и в худшем случае (х.с.).

Стандартный алгоритм: трудоемкость $O(13n^{3})$.

Алгоритм Винограда: л.с. -- $O(11.5n^{3})$,  х.с. --  $O(11.5n^{3})$.

Оптимизированный алгоритм Винограда: л.с. -- $O(11.5n^{3})$,  х.с. -- $O(11.5n^{3})$.