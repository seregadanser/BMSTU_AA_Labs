
\chapter*{Введение}
\addcontentsline{toc}{chapter}{Введение}

Матрицы --- крайне мощный инструмент, используемый в каждой точной науке: физике, математике, прораммировании и т.д.
Они позволяют выполнять операции, требующие большого количества входных параметров и сложные в вычислении, с относительной легкостью.
В связи с этим встает вопрос об оптимизации основных алгоритмов, связанных с матрицами: сложение, умножение, транспонирование и т.п.
В данной работе будет рассмотрена оптимизация алгоритмов матричного умножения.

Целью данной лабораторной работы является анализ трудоемкости реализаций алгоритмов матричного умножения.

Для достижения поставленной цели требуется решить ряд задач:
\begin{enumerate}
	\item[1)] изучить алгоритмы умножения матриц: классический, Винограда, оптимизированный Винограда;
	\item[2)] провести сравнительный анализ трудоёмкости алгоритмов на основе теоретических расчетов;
	\item[3)] реализовать каждый из трех алгоритмов;
	\item[4)] провести замеры процессорного времени для каждой из реализаций алгоритмов;
	\item[5)] выполнить анализ полученных результатов;
	\item[6)] по итогам работы составить отчет.
\end{enumerate}

\newpage
