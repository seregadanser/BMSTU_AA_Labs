
\chapter*{Введение}
\addcontentsline{toc}{chapter}{Введение}
\hyphenation{по-сле-до-ва-те-льно-сти}
В информатике сортировкой называется процесс организации последовательности в упорядоченном порядке.
Данная операция является одной из самых распространненых и важных в различных алгоритмах.  

На данный момент существует огромное количество вариаций сортировок.
Эти алгоритмы необходимо уметь сравнивать, чтобы выбирать наилучше подходящие в конкретном случае. 
Они оцениваются по следующим критериям:

\begin{itemize}
	\item времени быстродействия;
	\item затратам памяти.
\end{itemize}

Целью данной лабораторной работы является анализ трудоемкости реализаций алгоритмов сортировки.

Для достижения поставленной цели требуется решить следующие задачи.
\begin{enumerate}
	\item Изучить и реализовать алгоритмы сортировки: слиянием, плавная, блочная.
	\item Провести сравнительный анализ трудоёмкости алгоритмов на основе теоретических расчетов.
	\item Провести замеры процессорного времени работы реализаций выбранных сортировок.
	\item Провести сравнительный анализ реализаций алгоритмов по затраченному процессорному времени и памяти.
	\item Описать и обосновать полученные результаты в отчете о выполненной лабораторной работе, выполненного как расчётно-пояснительная записка к работе.
\end{enumerate}

\newpage
