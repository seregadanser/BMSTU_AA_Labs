\chapter{Исследовательская часть}

В данном разделе будут приведены примеры работы программы, а также проведен сравнительный анализ процессорного времени работы реализаций алгоритмов при различных ситуациях на основе полученных данных.

\section{Технические характеристики}

Технические характеристики устройства, на котором выполнялись замеры времени представлены далее:

\begin{itemize}
	\item операционная система Windows 11 Pro Версия 22H2 (22621.674) \cite{wind};
	\item память 16 ГБ;
	\item процессор 11th Gen Intel(R) Core(TM) i5-11400 2.59 ГГц \cite{proc}.
\end{itemize}

При тестировании компьютер был включен в сеть электропитания. Во время замеров процессорного времени устройство было нагружено только встроенными приложениями окружения, а также системой тестирования.

\section{Демонстрация работы программы}

На рисунке \ref{img:res} представлен результат работы программы. На экран выводятся результаты заемров времени для разных размеров матриц и разных видов алгоритмов матричного умножения в мс.
\newpage
%\includepdf[pages=-]{src/screen.pdf}
\begin{center}
	\centering{\includegraphics[trim=0 0 0cm 0cm bb=0 0 504 320]{src/screen}}
	\captionof{figure}{Пример работы программы}
	\label{img:res}
\end{center}

\section{Время выполнения реализаций алгоритмов}

Как было сказано выше, используется функция замера процессорного времени GetProcessTimes(...) из библиотеки Windows.h. 

\textbf{Входные данные:} линейная размерность квадратной матрицы от 10 до 500 для л.с., от 11 до 501 для х.с., элементы матрицы -- целые числа от 0 до 200.

Результаты замеров времени работы реализаций алгоритмов матричного умножения на различных входных данных (в мс) приведены в таблицах \ref{tbl:best}, \ref{tbl:best1}.


\begin{center}
	\begin{threeparttable}
		\caption{Процессорное время работы реализаций алгоритмов для четной размерности M}
		\label{tbl:best}
		\begin{tabular}{|c|c|c|c|}
			\hline
			Размер & Классический &  Винограда &  Оптимизированный\\
			\hline
			10 & 0  &  0.015625 &0  \\ 
			\hline
			20 &0.03125  &      0.03125 &      0.015625\\ 
			\hline
			30 &  0.0625        &0.03125  &      0.03125 \\ 
			\hline
			40 &  0.046875     &    0.0625 &        0.0625  \\ 
			\hline
			50 & 0.34375      &  0.28125    &   0.296875  \\ 
			\hline
			60 & 0.671875    &     0.4375    &     0.4375 \\ 
			\hline
			70 &  1.04688   &     0.65625     &   0.78125 \\ 
			\hline
			80 & 1.57812   &     1.15625       & 1.14062 \\ 
			\hline
			90 & 2.34375  &       1.6875        &  1.625 \\ 
			\hline
			100 & 3.07812&        2.23438        &2.23438 \\ 
			\hline
			200 & 25 &       18.0781      &  18.1406\\ 
			\hline
			300 &  85.7031    &    63.2656 &       61.4531 \\ 
			\hline
			400 &  206.312   &      150.75  &      150.516  \\ 
			\hline
			500 &  411.062  &      301.531   &     302.391  \\ 
			\hline
		\end{tabular}
		
	\end{threeparttable}
\newpage
\begin{threeparttable}
	\caption{Процессорное время работы реализаций алгоритмов для нечетной размерности M}
	\label{tbl:best1}
	\begin{tabular}{|c|c|c|c|}
		\hline
		Размер & Классический &  Оптимизированный Винограда  &  Винограда\\
		\hline
		11 &  0.015625 &      0.015625      & 0.015625 \\ 
		\hline
		21 &  0.0625    &   0.046875       &0.046875  \\ 
		\hline
		31 & 0.15625     &  0.109375        &  0.125  \\ 
		\hline
		41 &  0.34375     &  0.265625      & 0.296875  \\ 
		\hline
		51 & 0.65625       &0.484375      & 0.515625   \\ 
		\hline
		61 &  1.04688       &  0.8125    &    0.90625  \\ 
		\hline
		71 &  1.76562        & 1.3125   &     1.32812  \\ 
		\hline
		81 &   2.85938        &1.92188 &       1.95312 \\ 
		\hline
		91 & 3.54688   &      2.6875  &      2.70312   \\ 
		\hline
		101 & 4.71875   &     3.57812       & 3.67188  \\ 
		\hline
		201 &  37.7031   &     29.2188     &   29.6719  \\ 
		\hline
		301 & 145.453     &   108.859     &   111.656  \\ 
		\hline
		401 &  358.453     &   282.562   &     350.109 \\ 
		\hline
		501 &  810.516      &  590.203  &      636.422  \\ 
		\hline
	\end{tabular}
	
\end{threeparttable}
\end{center}

Также на рисунках \ref{img:graph_sorted} и \ref{img:graph_sorted1} приведены графические результаты замеров времени работы алгоритмов в зависимости от линейного размера входной матрицы $M$.

\begin{center}
	\centering{\includegraphics[trim=0 0 0 -5cm bb=0 0 570 650]{src/Graph}}
	\captionof{figure}{Процессорное время вычислений: четная размерность}
	\label{img:graph_sorted}
\end{center}
\newpage

\begin{center}
	\centering{\includegraphics[trim=0 0 0 -5cm bb=0 0 570 650]{src/Graph1}}
	\captionof{figure}{Процессорное время вычислений: нечетная размерность}
	\label{img:graph_sorted1}
\end{center}
\newpage


\section{Вывод}
%\addcontentsline{toc}{section}{Вывод}


Теоретические результаты оценки трудоемкости и полученные практически результаты замеров процессорного времени совпадают. Алгоритмы Винограда выполняются быстрее, чем стандартный алгорим умножения матриц, примерно в $1.3$ раза.  Различия между временами выполнения реализаций алгоритмов Винограда и стандартного алгоритма становятся более различимы при наступлении худшего случая по трудоёмкости для алгоритма Винограда --- при нечётном $N$.