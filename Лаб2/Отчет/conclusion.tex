\chapter*{Заключение}
\addcontentsline{toc}{chapter}{Заключение}

Цель, которая была поставлена в начале лабораторной работы, была достигнута: выполнен анализ трудоемкости реализаций алгоритмов матричного умножения.

В ходе выполнения лабораторной работы были решены все задачи:
\begin{enumerate}
	\item[1)] изучены алгоритмы умножения матриц --- классический, Винограда, оптимизированный Винограда;
\item[2)] проведен сравнительный анализ трудоёмкости алгоритмов на основе теоретических расчетов;
\item[3)] реализован каждый из трех алгоритмов;
\item[4)] проведены замеры процессорного времени для каждой из реализаций алгоритмов;
\item[5)] выполнен анализ полученных результатов;
\item[6)] по итогам работы составлен отчет.
\end{enumerate}

В ходе проделанной работы было выявлено, что различия в процессорном времени выполнения двух версий алгоритма Винограда не являются существенными и не превосходят разрыва в $0.9$ раз. В это же время стандартный алгоритм заметно отстает во времени выполнения при увелечении размеров матрицы. Из полученных данных видно, что время работы реализаций алгоритмов возрастает  кубически с увеличением линейной размерности матриц.
