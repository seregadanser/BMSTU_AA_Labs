\chapter{Технологическая часть}
В данном разделе будут рассмотрены средства реализации, а также представлены листинги сортировок.
\section{Средства реализации}
В данной работе для реализации был выбран язык программирования $c++$. В текущей лабораторной работе требуется замерить процессорное время для выполняемой программы, а также построить графики. Для визуализации результатов использовался язык $Python$.

Время работы было замерено с помощью функции \textit{GetProcessTimes(...)} \cite{time} из библиотеки $Windows.h$.

\section{Сведения о модулях программы}
Программа состоит из следующих модулей:
\begin{itemize}
	\item $AA2.cpp$ - файл, содержащий весь служебный код;
	\item $MatrixMull.cpp$ - файл, содержащий код всех алгоритмов матричного умножения;
	\item $TimeCompare.cpp$ - файл, производящий замеры времени;
	\item $GetCPUTime.cpp$ - файл, определяющий функцию замера времени;
	\item $Gen.cpp$ - файл, генерирующий входную матрицу;
	\newpage
	\item $MatrixMull.hpp$ - заголовочный файл модуля $MatrixMull.cpp$;
	\item $TimeCompare.hpp$ - заголовочный файл модуля $TimeCompare.cpp$;
	\item $GetCPUTime.hpp$ - заголовочный файл модуля $GetCPUTime.cpp$;
	\item $Gen.hpp$ - заголовочный файл модуля $Gen.cpp$.\newline
\end{itemize}

\section{Реализация алгоритмов}
В листингах \ref{standart}, \ref{vino}, \ref{vino_o}, представлены реализации алгоритмов матричного умножения (стандартный, винограда, оптимизированный винограда). 
\begin{center}
\begin{lstlisting}[label=standart, caption={Стандартный алгоритм умножения матриц}]
data_t matrix_multiplication(const data_t m1, const data_t m2)
{
	data_t result;
	result.n = m1.n;
	result.m = m2.m;
	result.matrix = create_matrix(m1.n, m2.m);
	if (!result.matrix)
	return result;
	for (int i = 0; i < m1.n; i++)
	for (int j = 0; j < m2.m; j++)
	{
		result.matrix[i][j] = 0;
		for (int k = 0; k < m1.m; k++)
		result.matrix[i][j] += m1.matrix[i][k] * m2.matrix[k][j];
	}
	return result;
}
\end{lstlisting}
\end{center}
\newpage
\begin{center}
\begin{lstlisting}[label=vino, caption={Алгоритм Винограда умножения матриц}]
data_t matrix_multiplication_Vinograd(const data_t m1, const data_t m2)
{
	data_t result;
	result.n = m1.n;
	result.m = m2.m;
	result.matrix = create_matrix(m1.n, m2.m);
	if (!result.matrix)
	return result;
	int n = m1.n;
	int m = n;
	
	int *rowFactor = (int *)calloc(n, sizeof(int)), *colFactor = (int*)calloc(n, sizeof(int));
		for (int i = 0; i < n; ++i)
	{
		rowFactor[i] = m1.matrix[i][0] * m1.matrix[i][1];
		for (int j = 1; j < d; ++j)
		rowFactor[i] +=  m1.matrix[i][j<<1] * m1.matrix[i][(j<<1) + 1];
	}
	
	for (int i = 0; i < n; ++i)
	{
		colFactor[i] = m2.matrix[0][i] * m2.matrix[1][i];
		for (int j = 1; j < d; ++j)
		colFactor[i] += m2.matrix[j<<1][i] * m2.matrix[(j<<1) + 1][i];
	};
	
	for (int i = 0; i < n; ++i)
	for (int j = 0; j < n; ++j)
	{
		result.matrix[i][j] = -(rowFactor[i] + colFactor[j]);
		for (int k = 0; k < d; ++k)
		{
			result.matrix[i][j] += (m1.matrix[i][k << 1] + m2.matrix[(k<<1) + 1][j]) * (m1.matrix[i][(k<<1) + 1] + m2.matrix[ k<<1][j]);
		}
	}
	
	if (m % 2 != 0)
	{
		for (int i = 0; i < n; ++i)
		for (int j = 0; j < n; ++j)
		{
			result.matrix[i][j] += m1.matrix[i][m - 1] * m2.matrix[m - 1][j];
		}
	}

	free(rowFactor);
	free(colFactor);
	return result;
}
\end{lstlisting}
\end{center}
\newpage
\begin{center}
\begin{lstlisting}[label=vino_o,caption = {Оптимизированный алгоритм Винограда умножения матриц}]
data_t matrix_multiplication_VinogradOptimase(const data_t m1, const data_t m2)
{
	data_t result;
	result.n = m1.n;
	result.m = m2.m;
	result.matrix = create_matrix(m1.n, m2.m);
	if (!result.matrix)
	return result;
	int n = m1.n;
	int m = n;
	int d = n>>1;
	int *rowFactor = (int *)calloc(n, sizeof(int)), *colFactor = (int*)calloc(n, sizeof(int));
		for (int i = 0; i < n; ++i)
	{
		for (int j = 0; j < m / 2; ++j)
		rowFactor[i] = rowFactor[i] + m1.matrix[i][2 * j] * m1.matrix[i][2 * j + 1];
	}
	
	for (int i = 0; i < n; ++i)
	{
		for (int j = 0; j < m / 2; ++j)
		colFactor[i] = colFactor[i] + m2.matrix[2 * j][i] * m2.matrix[2 * j + 1][i];
	};
	
	for (int i = 0; i < n; ++i)
	for (int j = 0; j < n; ++j)
	{
		result.matrix[i][j] = -(rowFactor[i] + colFactor[j]);
		for (int k = 0; k < m / 2; ++k)
		{
			result.matrix[i][j] = result.matrix[i][j] + (m1.matrix[i][2 * k] + m2.matrix[2 * k + 1][j]) * (m1.matrix[i][2 * k + 1] + m2.matrix[2 * k][j]);
		}
	}
	
	if (m % 2 != 0)
	{
		for (int i = 0; i < n; ++i)
		for (int j = 0; j < n; ++j)
		{
			result.matrix[i][j] = result.matrix[i][j] + m1.matrix[i][m - 1] * m2.matrix[m - 1][j];
		}
	}
	free(rowFactor);
	free(colFactor);
	return result;
}
\end{lstlisting}

\end{center}
\newpage
\section{Функциональные тесты}

В таблице \ref{tbl:functional_test} приведены тесты для функций, реализующих алгоритмы матричного умножения. Тесты для всех реализаций алгоритмов пройдены успешно.
	\begin{center}
		\begin{threeparttable}
				\caption{\label{tbl:functional_test} Функциональные тесты}
		\begin{tabular}{|c@{\hspace{7mm}}|c@{\hspace{7mm}}|c@{\hspace{7mm}}|c@{\hspace{7mm}}|c@{\hspace{7mm}}|c@{\hspace{7mm}}|}
			\hline
			Матрица 1 & Матрица 2 & Ожидаемый результат \\ 
			\hline
			
			$\begin{pmatrix}
			1
			\end{pmatrix}$ &
			$\begin{pmatrix}
				&
			\end{pmatrix}$ &
			Сообщение об ошибке \\ \hline
			
			$\begin{pmatrix}
				1 & 2
			\end{pmatrix}$ &
			$\begin{pmatrix}
				1 & 2 & 3
			\end{pmatrix}$ &
			Сообщение об ошибке \\ \hline
			
			$\begin{pmatrix}
				1 \\
				1 \\
				1
			\end{pmatrix}$ &
			$\begin{pmatrix}
				1 & 1 & 1
			\end{pmatrix}$ &
			$\begin{pmatrix}
				1 & 1 & 1\\
				1 & 1 & 1 \\
				1 & 1 & 1
			\end{pmatrix}$ \\ \hline
			
			$\begin{pmatrix}
				1 & 1 & 1 \\
				2 & 2 & 2 \\
				3 & 3 & 3
			\end{pmatrix}$ &
			$\begin{pmatrix}
				1 & 0 & 0 \\
				0 & 1 & 0 \\
				0 & 0 & 1
			\end{pmatrix}$ &
			$\begin{pmatrix}
				1 & 1 & 1 \\
			2 & 2 & 2 \\
			3 & 3 & 3
			\end{pmatrix}$ \\ \hline
			
			$\begin{pmatrix}
				1
			\end{pmatrix}$ &
			$\begin{pmatrix}
				2
			\end{pmatrix}$ &
			$\begin{pmatrix}
				2
			\end{pmatrix}$ \\ \hline
			
		\end{tabular}
	
		\end{threeparttable}
	\end{center}



\section*{Вывод}
% \addcontentsline{toc}{section}{Вывод}
В данном разделе были представлены реализации следующих алгоритмов: стандартного матричного умножения, Винограда, Винограда с оптимизацией.